\section{Structural units of LIGNUM}
LIGNUM is intended as a generic modelling tool for both coniferous and
hardwood trees.  Different tree  species can be simulated by assessing
descriptions of  metabolism, structural dynamics of  birth, growth and
senescence,  and by  implementing  branching rules  for distinct  tree
architectures  \citep{perttunen:96,  perttunen:01}.   Here we  present
shortly the main model features to understand how LIGNUM is adapted to
use L-systems.

LIGNUM represents the three-dimensional  above ground part of the tree
with  four  structural  units  called  tree  segment  (TS),  bud  (B),
branching point  (BP) and  axis (A).   A branching point  is a  set of
axes. An  axis is  a sequence of  tree segments, branching  points and
terminating bud.   This captures the  recursive structure of  the tree
crown.

The most  important functioning unit is the  cylindrical tree segment,
section of woody material  between two branching points.  For conifers
needles are  at the  moment modeled as  cylindrical layers  of foliage
surrounding tree segments (Fig.  \ref{fig:model} left).  For deciduous
species leaves  attached to tree  segments are considered  and studied
explicitly  using simple  geometric form  like ellipse  as  leaf shape
(Fig. \ref{fig:model}, middle).

An axis  is implemented as a  list. For example  adopting the notation
from our previous work \citep{perttunen:96} the main axis of the model
tree for  coniferous species  in Figure \ref{fig:model}  consisting of
three tree  segments, three branching  points and the  terminating bud
writes:

\begin{equation}
[TS_0,BP_1,TS_2,BP_3,TS_4,BP_5,B_6]
\end{equation}

A branching  point is implemented as  list of axes. In  the example of
Figure   \ref{fig:model}  the  branching   points  contain   two  axes
(i.e. branches) each. Thus the main axis can be written:

\begin{equation}\label{eq:mainax}
[TS_0,[[A,A]],TS_2,[A,A],TS_4,[A,A],B_6]
\end{equation}

The branches of the example  tree in Figure \ref{fig:model} consist of
tree  segment, branching  point and  bud. When  these are  inserted in
Eq. \ref{eq:mainax},  the structure  of  the whole  model  tree can  be
expressed as (omitting the position numbering):

\begin{equation}\label{eq:tree}
[TS,[[TS,[],B],[TS,[],B]],TS,[[TS,[],B],[TS,[],B]],TS,[[B],[B]],B]
\end{equation}

The empty lists  ([]) for branching points denote  no axes forking off
maintaining the structural integrity of the model.

