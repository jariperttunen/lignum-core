\section{Structural units of LIGNUM}
LIGNUM is intended as a generic modelling tool for both coniferous and
hardwood trees.  Different tree  species can be simulated by assessing
descriptions of  metabolism, structural dynamics of  birth, growth and
senescence,  and by  implementing  branching rules  for distinct  tree
architectures.  The  general framework  of LIGNUM for  both coniferous
and hardwood  species has been presented  in [Perttunen et  al 96] and
[Perttunen et al 2001] respectively.  Here we present shortly the main
model features to understand how LIGNUM is adapted to use L-systems.

LIGNUM represents the three-dimensional  above ground part of the tree
with  four  structural  units  called  tree  segment  (TS),  bud  (B),
branching point (BP) and axis (A). The most important functioning unit
is the cylindrical tree segment, section of woody material between two
branching points.   For conifers  needles are modelled  as cylindrical
layers of  foliage surrounding  tree segments.  For  deciduous species
leaves attached  to tree segments are  modelled explicitly. Branching
point connects one or more tree  segments and an axis is a sequence of
tree segments and branching points terminating in a bud.

An axis  is implemented as  a list. For  example the main axis  of the
model  tree for  deciduous species  consists of  four  tree segments,
three branching points and the terminating bud:

\begin{equation}
[TS_0,BP_1,TS_2,BP_3,TS_4,BP_5,B_6]
\end{equation}

A branching point is implemented as list  of axes. In the model tree the
first branching  point contains one axis  with one bud  and the second
and the third two axes each. Thus the main axis can be written:

\begin{equation}
[TS_0,[[B]],TS_2,[A,A],TS_4,[A,A],B_6]
\end{equation}

Finally, the structure of the whole model tree is expressed as:

\begin{equation}
[TS_0,[[B]],TS_2,[[TS,[],B],[TS,[],B]],TS_4,[[B],[B]],B_6]
\end{equation}

The two empty  lists ([]) for branching points  denote no axes forking
off maintaining the structural integrity of the model.

So  far the  root system  in LIGNUM  has been  modelled with  a single
parameter denoting  its mass.  [Siev�nen et al  00] discuss components
of the functional-structural tree  models, both above ground and below
ground part of the tree.  The proposed idealized elementary unit (IEU)
for root  system is  a internode between  two branching  events having
descriptions for  the three-dimensional (woody)  structure, metabolism
and transport  processes.  

We  implement these  ideas  slightly modified  by defining  structural
elements and the branching structure for roots in LIGNUM very much the
same  compared with  above ground  part of  the tree.   The structural
elements are  root segment (RS),  root tip (RT), root  branching point
(RBP) and root axis (RA) corresponding to tree segment, bud, branching
point and axis respectively.   The root segment consists of heartwood,
sapwood and bark  and can have a cylindrical layer  of fine roots.  In
this  manner we  believe  the  structural units  are  small enough  to
capture realistic  three dimensional structure of the  root system but
large enough to keep the number of units computationally tolerable.

