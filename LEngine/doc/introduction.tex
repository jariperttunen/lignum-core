\section{Introduction} 

An increasing number of models  (see e.g. Annals of Forest Science vol
57 no.  5/6) try to depict  the dynamics and growth of woody perennial
plants  by  assessing  the  physiological  processes  in  their  three
dimensional  arborescent form.   Physiological processes  involve, for
example, photosynthesis  of the  foliage, respiration, flow  of water,
nutrients or hormones and  the allocation of nutrients.  The structure
of the tree  and its changes over time are  described for example with
state variables representing different aggregated tree compartments or
with  detailed  modelling  components  faithful to  the  botanical  or
morphological  units   of  plants.   Such  models   have  been  called
functional-structural tree models (FSTM).

\citet{kurth:94b} has  classified tree and forest models  on the basis
of  whether  the  emphasis is  on  the  structural  traits or  on  the
functioning of trees.  Accordingly,  there are principally two ways to
construct  a  FSTM.   One   can  start  from  an  architectural  model
\citep{jaeger:92,  kurth:94} and  add functional,  i.e., physiological
detail to  it.  The second approach  is to begin with  a process-based
physiological  model \citep{makela:86, landsberg:86,  sievanen:93} and
extend it with structural details.

Linking these  two kinds of  model together raises the  question about
the  most suitable way  of expressing  them. Physiological  models are
usually realized with the  aid of differential or difference equations
\citep{landsberg:86}  whereas architectural  models  apply Lindenmayer
systems  \citep{kurth:99,pp:90}  or  other  formalisms. Would  a  FSTM
combine those means in fact or rely on either of them?

The LIGNUM model approaches FSTM  from the physiological side; it is a
single tree model  \citep{perttunen:96} with fidelity to process-based
modelling  \citep[see  e.g.][]{nikinmaa:92, sievanen:93,  makela:97-1}
but,  instead of  aggregated tree  parts, it  has  a three-dimensional
description of the  above ground part of the  tree.  LIGNUM includes a
detailed    model    of    self-shading    within   a    tree    crown
\citep{perttunen:96,  perttunen:01}, from  which the  radiation regime
for photosynthesis in different parts of the tree can be computed.  If
the photosynthates produced exceed  the respiration costs then the net
production is allocated  to the growth of new  parts.  LIGNUM has been
applied to both coniferous \citep{perttunen:96,lo:99} and broad-leaved
trees \citep{perttunen:01}.  The main focus  in LIGNUM has been on the
functional  part of  the  model and  less  emphasis has  been paid  to
structural development.  The model  does not include any formal method
to define the architectural development of the tree structure.

Since  the pioneering  work  by \citet{honda:71}  methods have  become
available for  treating plant architecture  in models in  an efficient
way.  Lindenmayer  systems, or L systems for  short \citep{pp:89}, are
the most widely used method to treat plant architecture although other
formalisms   also  exist  \citep[e.g.][]{dereffye:97,   godin:99}.   L
systems    were    invented    by    Aristid    \citet{lindenmayer:68,
  lindenmayer:71} and were initially meant to describe the development
of  multicellular organisms.  L  systems are  string-rewriting systems
and research on  these systems is concerned with  the question of what
phenomena can  be described with formal languages.   The theory, tools
and  applications that  utilize  a L  system  framework for  modelling
plants  and their  environment  have been  developed by  Prusinkiewicz
\citep{pp:89,pp:92}, Kurth \citep{kurth:94} and other scientists.  The
ubiquitous  theory and  the progress  made in  L systems  in modelling
plants and trees  up until the end of the '90's  is well documented in
\citet{pp:90,pp:99} and \citet{kurth:99}.

We describe in the following  how the LIGNUM model has been interfaced
with   L  systems   for   specifying  formally   and  rigorously   the
architectural   development   of    trees,   thereby   improving   the
applicability and ease  of use of this FSTM.  The  goal is achieved by
using  the   L  language,   which  is  an   extension  of   L  systems
\citep{pp:99a}.  Based  on the definition of L,  R.  Karwowski created
the  original  parser  of  L  and has  implemented  the  L+C  language
\citep{karwowski:02}.  He  also   included  further  improvements  not
present in  L, most notably  productions with multiple  successors and
the concept  of new context \citep{karwowski:03} to  allow fast linear
time information transfer in the simulated plant.

We first present the use of L in LIGNUM based on the similarity of how
LIGNUM and bracketed L systems represent branching structures of trees
\citep{perttunen:96, perttunen:01}.  Second, we implement an algorithm
that translates  the bracketed  string of symbols  in L to  the LIGNUM
representation of trees (Section \ref{sec:pine}).  Third, we provide a
communication  mechanism  between the  L  string  and LIGNUM  (Section
\ref{sec:LToLignum}), and give two example applications, one for Scots
pine  and the  other for  bearberry  (\textit{Arctostaphylos uva-ursi}
L.).  We present  a simple language construct that  allows us to model
groups   of   plants   each   with   its   own   L   system   (Section
\ref{sec:namespace}).   Finally, we  discuss the  consequences  of our
approach for the  future development of LIGNUM in  modelling trees and
forest stands.

