\section{Introduction}
Functional-structural tree models (FSTM) try to determine dynamics and
growth of woody perennial  plants by assessing physiological processes
in their three  dimensional arborescent form.  Physiological processes
involve for  example photosynthesis of the  foliage, respiration, flow
of  water,   nutrients  or   hormones  and  allocation   of  nutritive
substances.   The structure of  the tree  and its  changes in  time is
described  for  example with  state  variables representing  different
aggregated  tree compartments  or with  detailed  modelling components
faithful to botanical or morphological units of plants.

Kurth  has constructed  a model  triangle  \citep{kurth:94b} arranging
existing modelling approaches in  forestry into three categories based
on the  emphasis in the modeling approaches  (structure vs.  function,
single tree vs.  forest stand).  Accordingly there are practically two
ways to construct an FSTM  \citep{sievanen:00}.  One can start from an
architectural  model   \citep{jaeger:92,kurth:94}  adding  functional,
physiological details into it.  The second approach is to begin with a
process  based  physiological  model  \citep{makela:86,  landsberg:86,
  sievanen:93} and extend it with structural details.

LIGNUM  is  a  single  tree  model  \citep{perttunen:96}  faithful  to
process-based  modelling \citep[see  e.g.][]{nikinmaa:92, sievanen:93,
  makela:97-1}  but instead  of  aggregated tree  parts  it has  three
dimensional  description  of  the  above  ground  part  of  the  tree.
Following the principles in \citet{hari:82} LIGNUM includes a detailed
model  of  self-shading   within  a  tree  crown  \citep{perttunen:96,
  perttunen:01} from which the  radiation regime for photosynthesis in
different parts  of the tree  can be computed.  If  the photosynthates
produced exceed  respiration costs the net production  is allocated to
new and existing tree  structure acquiring carbon balance.  LIGNUM has
been applied for  both coniferous \citep{perttunen:96,lo:99} and broad
leaved trees \citep{perttunen:01}.

The main  focus in LIGNUM has  been to develop the  functional part of
the model.  Until now there has not been any method to formally define
the  architecural development  of  the tree  structure.  Research  has
produced  remarkable  results  in  this  field  however.   

A leaf  of Barnsley's fern \citep{barnsley:00} is  a prominent example
of fractal theory  providing a way to describe  the self similarity of
natural  objects.  \citet{gielis:03}   has  found  a  single  formula,
essentially a generalized equation  of ellipse, that can describe many
forms and shapes of natural objects differing in parameter values only.

Hall\'e  \citep{halle:78} has studied  tropical botany,  analysed tree
structure  and produced  25  architectural models  based on  branching
pattern, senescence and functioning  of buds.  Based on Halle's models
de  Reffye   \citep[overview][]{dereffye:89,  jaeger:92,  dereffye:95,
  dereffye:97} has developed  botanically sound automaton theory which
has  been   put  into  practice   in  the  family  of   AMAP  software
\citep{fourcard:97}.  de  Reffye's automaton theory  have been further
developed by  \citet{yan:01} where a plant's  topological structure is
composed   from  its   subparts  avoiding   possibly   repetitive  and
computatinally costly internode by internode construction of trees.

Godin \citep{godin:99} has presented  a method for describing topology
and geometry of  plants in several scales with  multiscale tree graphs
(MTG).  Formally an  MTG  is  a set  of  tree graphs  \citep{godin:98}
capturing different levels of details in a plant.

Lindenmayer-systems (L-systems)  invented 1968 by  Aristid Lindenmayer
\citep{lindenmayer:68,   lindenmayer:71}  were   initially   meant  to
describe  the development of  multicellular organisms.   L-systems are
string  rewriting  systems  and   their  research  is  concerned  what
phenomena can  be described with  formal languages. The  theory, tools
and  applications using  L-systems framework  for modeling  plants and
their    environment    have    been   developed    by    Prusinkewicz
\citep{pp:89,pp:92}, Kurth \citep{kurth:94} and other scientists.  The
ubiquitous theory  and the progress  in L-systems in  modelling plants
and   trees  until   the  end   of   '90's  is   well  documented   in
\citet{pp:96,pp:99} and \citet{kurth:99}.

The  objective of this  study is  to interface  the model  LIGNUM with
L-systems  so  that the  architectural  development  of  trees can  be
specified formally  and rigorously. This is accomplished  by using the
language L [Prusinkiewicz and  Karwowski??].  We first present the use
of  L in  LIGNUM  based on  the  similarity how  LIGNUM and  bracketed
L-systems     represent      branching     structures     of     trees
\citep{perttunen:96,perttunen:01}.  Second,  we implement an algorithm
that   translates  the   bracketed   string  of   symbols  to   LIGNUM
representation of  trees (Section \ref{sec:pine}).   Third, we provide
communication  mechanism  between  the  L-string and  LIGNUM  (Section
\ref{sec:bearberry}).  Finally,  we present simple  language construct
that allows  us to model  group of plants  each with its  own L-system
(Section  \ref{sec:namespace}).   We   give  sample  applications  and
discuss our approach for its consequences to the future development of
LIGNUM in modeling trees and forest stands.

