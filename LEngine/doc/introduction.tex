\section{Introduction}

Functional-structural tree models (FSTM) try to determine dynamics and
growth of woody perennial  plants by assessing physiological processes
(function)  in their three  dimensional arborescent  form (structure).
Physiological  processes  or  plant  metabolism involves  for  example
photosynthesis of  the foliage  using the radiation  regime encircling
the  plant, respiration,  flow  of water,  nutrients  or hormones  and
allocation of  nutritive substances.  The  three dimensional structure
of the  tree and its  changes in time  should be modelled in  terms of
detailed  botanical or  morphological  units rather  than with  highly
aggregated plant compartments.

The  term 'functional-structural  model' was  introduced  in [Silva97]
that compiles  a series of  articles where the  focus was set  on what
happens in  a single  plant element (such  as shoot,  internode, leaf,
root hairs etc.)  and  defining their interactions.  Obviously, due to
possibly thousands of interacting units such models are implemented as
computer programs where the program acts as the model and as the means
to follow,  solve and analyse the  model behaviour (\textit{In silico}
simulations).

In  Montpellier, a  city with  over  10 centuries  of known  botanical
heritage, the modelling group AMAP  at CIRAD ('French school') has the
basis for  their plant growth  and analysis systems  AMAPsim, AMAPpara
and AMAPmod  in the architectural analysis by  Halle\'{e} grouping the
morphological caharacteristics of tropical trees into 25 model classes
and the notion  of reiteration observed by Oldeman  where a plant body
is able to duplicate its existing structure.

AMAP re-introduced  the notion of  physiological age as  distinct from
real age  to organize the irreversible stochastic  processes of birth,
growth, branching and death of an apical bud (meristem). Each stage of
the bud  from birth to  death can be  arranged into a  theoretical one
dimensional reference  axis (\textit {axe  de r\'{e}f\'{e}rence}) with
shifts  from one  stage to  another as  possible paths  to  modify the
physiological age of  the bud.  The associated finite  automaton has a
state for each stage in  the reference axis, a transition function for
the shifts between the stages and additional geometric information for
branching and  growth. This finite  automaton system is  sufficient to
determine    architectural   tree   growth    including   reiteration,
self-pruning   and   changes  in   form   as   tree  develops   called
metamorphosis.  The  reference axis  was fully implemented  in AMAPsim
used  to produce  realistic images  of important  trees and  plants in
forestry and agriculture.



Examples

-AMAP 'French School'

-'L-System School'

-'Helsinki School'

extending the  traditional three  step analysis of  a plant  where the
nature  of the  intrinsic processes  of the  entity are  analysed, the
differential  equations  describing  the  processes are  inferred  and
solved possibly  assuming some reasonably sufficient  geometry for the
entity.

By  looking  The  two  perspectives  of  plant  modelling,  structural
dynamics  and  physiological  processes,  have  evolved  independently
simplifying  or  simply  neglecting   the  standpoint  of  the  other.
However, the progress  in these two lines of  modelling have gradually
involved characteristics of each other bringing them closer.



LIGNUM is an FSM applied to 
Current interest

Objectives of this study.
 
Winfried  Kurth  has   classified  existing  modelling  approaches  in
forestry  into   three  categories  and  based   on  this  observation
constructed a model triangle.  The  top apex of this triangle is taken
up by aggregated  stand models.  Process based single  tree models and
architectural tree models  are placed at the base  apices of the model
triangle.  Moving along the sides and inside the triangle captures the
model continuum including aspects from  one or the other two modelling
avenues. 

