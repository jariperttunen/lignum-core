\section{Introduction}
Functional-structural tree models (FSTM) try to determine dynamics and
growth of woody perennial  plants by assessing physiological processes
in their three  dimensional arborescent form.  Physiological processes
involve for  example photosynthesis of the  foliage, respiration, flow
of  water,   nutrients  or   hormones  and  allocation   of  nutritive
substances.   The structure of  the tree  and its  changes in  time is
described  for  example with  state  variables representing  different
aggregated  tree compartments  or with  detailed  modelling components
faithful to botanical or morphological units of plants.

Kurth  has constructed  a model  triangle  \citep{kurth:94b} arranging
existing modelling approaches in  forestry into three categories based
on the  emphasis in the modeling approaches  (structure vs.  function,
single tree vs.  forest stand).  Accordingly there are practically two
ways to construct an FSTM  \citep{sievanen:00}.  One can start from an
architectural  model   \citep{jaeger:92,kurth:94}  adding  functional,
physiological details into it.  The second approach is to begin with a
process  based  physiological  model  \citep{makela:86,  landsberg:86,
  sievanen:93} and extend it with structural details.

LIGNUM  is  a  single  tree  model  \citep{perttunen:96}  faithful  to
process-based  modelling \citep[see  e.g.][]{nikinmaa:92, sievanen:93,
  makela:97-1}  but instead  of  aggregated tree  parts  it has  three
dimensional  description  of  the  above  ground  part  of  the  tree.
Following the principles in \citet{hari:82} LIGNUM includes a detailed
model  of  self-shading   within  a  tree  crown  \citep{perttunen:96,
  perttunen:01} from which the  radiation regime for photosynthesis in
different parts  of the tree  can be computed.  If  the photosynthates
produced exceed  respiration costs the net production  is allocated to
new and existing tree  structure acquiring carbon balance.  LIGNUM has
been applied for  both coniferous \citep{perttunen:96,lo:99} and broad
leaved trees \citep{perttunen:01}.

The main  focus in LIGNUM has  been to develop the  functional part of
the model;  the model has not  included any method  to formally define
the  architectural development  of the  tree structure.   Research has
produced  remarkable  results  in  this field  however  starting  from
pioneering work  by \citet{honda:71}.  He studied the  crown shapes of
trees  using  simple but  botanically  based  deterministic rules  for
branching  angles, number  of  branches produced  and branch  lengths.
Even with  such limited information for  branching Honda's simulations
could produce  distinct tree-like  shapes simply by  changing slightly
numerical parameters in the rules.

In  their  capital  work  \citet{halle:78}  studied  tropical  botany,
analysed tree structure and established the concept tree architecture,
a morphological  expression of the genetic program  or blueprint which
determines the tree growth.  They recognized 23 distinct architectural
models found in tropical and  temperate trees based mainly on criteria
related to extension growth and branching dynamics.

The  work of  \citet{halle:78} is  the  sound botanical  basis for  de
Reffye's automaton theory \citep[see e.g.][]{dereffye:89, dereffye:95,
  dereffye:97, jaeger:92} which  has been put  into practice  in the
family  of AMAP software  \citep{fourcard:97}.  de  Reffye's automaton
theory have  been further developed by \citet{yan:01}  where a plant's
topological structure is composed  from its subparts avoiding possibly
repetitive   and   computatinally   costly  internode   by   internode
construction of trees.

Godin \citep{godin:99} has presented  a method for describing topology
and geometry of  plants in several scales with  multiscale tree graphs
(MTG).   Formally an  MTG is  a  set of  tree graphs  \citep{godin:98}
capturing different levels of details in a plant in a single model.

Lindenmayer-systems (L-systems)  invented 1968 by  Aristid Lindenmayer
\citep{lindenmayer:68,   lindenmayer:71}  were   initially   meant  to
describe  the development of  multicellular organisms.   L-systems are
string  rewriting  systems  and   their  research  is  concerned  what
phenomena can  be described with  formal languages. The  theory, tools
and  applications using  L-systems framework  for modeling  plants and
their    environment   have    been    developed   by    Prusinkiewicz
\citep{pp:89,pp:92}, Kurth \citep{kurth:94} and other scientists.  The
ubiquitous theory  and the progress  in L-systems in  modelling plants
and   trees  until   the  end   of   '90's  is   well  documented   in
\citet{pp:90,pp:99} and \citet{kurth:99}.

The  objective of this  study is  to interface  the model  LIGNUM with
L-systems  to  specify   formally  and  rigorously  the  architectural
development of trees.   This is accomplished by using  the language L.
The original specification of L  was by \citet{pp:99a}.  Based on this
work Karwowski  created the original  parser of L and  has implemented
the language  L+C \citep{karwowski:02} including  further improvements
not present  in L, most  notably productions with  multiple successors
and  the concept  of new  context \citep{karwowski:03}  to  allow fast
linear time information transfer in the simulated plant.

We first  present the use of L  in LIGNUM based on  the similarity how
LIGNUM and bracketed L-systems represent branching structures of trees
\citep{perttunen:96,perttunen:01}.  Second,  we implement an algorithm
that  translates  the bracketed  string  of  symbols  in L  to  LIGNUM
representation of  trees (Section \ref{sec:pine}).   Third, we provide
communication  mechanism  between  the  L-string and  LIGNUM  (Section
\ref{sec:bearberry}).  Finally,  we present simple  language construct
that allows  us to model  group of plants  each with its  own L-system
(Section  \ref{sec:namespace}).   We   give  sample  applications  and
discuss our approach for its consequences to the future development of
LIGNUM in modeling trees and forest stands.

