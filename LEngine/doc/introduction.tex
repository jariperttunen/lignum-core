\section{Introduction} 

An increasing number of models  (see e.g. Annals of Forest Science vol
57 no.  5/6) try to depict  the dynamics and growth of woody perennial
plants  by  assessing  the  physiological  processes  in  their  three
dimensional  arborescent form.   Physiological processes  involve, for
example, photosynthesis of the  foliage, respiration, flow of water or
hormones and the  allocation of nutrients.  The structure  of the tree
and  its  changes  over   time  are  described  with  state  variables
representing different  aggregated tree compartments  or with detailed
modelling components faithful to  the botanical or morphological units
of plants.   Such models  have been called  functional-structural tree
models (FSTM).

W.  \citet{kurth:94b}  has  classified tree and  forest  models on the
basis of whether  the emphasis is on  the structural traits or  on the
functioning of trees.  Accordingly, there  are principally two ways to
construct  a  FSTM.     One can start   from   an  architectural model
\citep{jaeger:92,  kurth:94} and  add  functional, i.e., physiological
detail to it.  The second  approach is to  begin with a  process-based
physiological model \citep{makela:86, landsberg:86, sievanen:93}   and
extend it with   structural details. Physiological models  are usually
realized with  the  aid of    differential  or difference    equations
\citep{landsberg:86} whereas  architectural  models  apply Lindenmayer
systems \citep{kurth:99,pp:90} or other formalisms.

Since  the pioneering  work  by \citet{honda:71}  methods have  become
available for  treating plant architecture  in models in  an efficient
way.  Lindenmayer  systems, or L systems for  short \citep{pp:89}, are
the most widely used method to treat plant architecture although other
formalisms   also  exist  \citep[e.g.][]{dereffye:97,   godin:99}.   L
systems    were    invented    by    Aristid    \citet{lindenmayer:68,
  lindenmayer:71} and were initially meant to describe the development
of  multicellular organisms.  L  systems are  string-rewriting systems
and research on  these systems is concerned with  the question of what
phenomena can  be described with formal languages.   The theory, tools
and  applications that  utilize  a L  system  framework for  modelling
plants  and their  environment  have been  developed by  Prusinkiewicz
\citep{pp:89,pp:92}, Kurth \citep{kurth:94} and other scientists.  The
ubiquitous  theory and  the progress  made in  L systems  in modelling
plants and trees  up until the end of the '90's  is well documented in
\citet{pp:90,pp:99} and \citet{kurth:99}.


All  process-based  models  \citep{landsberg:86}  for tree  and  stand
growth  must subsume  a  notion  for crown  or  canopy structure  that
matches  the objectives  of the  modelling.  A  facile solution  is to
assume  horizontally  homogeneous layers  of  foliage  used in  mainly
theoretical     studies     of      tree     and     forest     growth
\citep{hari:82,sievanen:93}.  Models that captured the individual stem
structure or partitioned  the above ground part of  the tree into even
finer  compartments  of  branches,  shoots  etc.  \citep{kellomaki:95,
  makela:97-1}  expanded the  scope  of the  process-based models  for
example to wood  quality applications \citep{kellomaki:99, makela:03}.
Though the traits  of physiology has been taken  into consideration in
detail,  the tree  architecture has  been  treated in  the same  model
varying in  detail and  the way  it has been  embedded in  the program
implementing the  model.  Hence each architecural model  is unique and
there  exist  no straightforward  way  of  comparing  these models  as
regards of architecture.


The methods  for plant architecture \citep{pp:90,dereffye:97,godin:99}
offer means for treating tree  architecture in a formal, and therefore
comparable way. However, these formalisms  have so far not matched the
capabilities  of general  programming languages  to deal  with diverse
programming  tasks (e.g.   modelling  diffusion of  substrates). In  a
number of model or modelling tools the suitable combination of methods
to  deal effectively  with both  architecure and  physiology  has been
addressed \citep{kurth:99, eschenbach:00, karwowski:03, yan:04}.  The
development of  models where the  architectureand functioning interact
is of key importance  for better understanding the structural dynamics
of trees.

The LIGNUM model approaches FSTM  from the physiological side; it is a
single tree model  \citep{perttunen:96} with fidelity to process-based
modelling  \citep[see  e.g.][]{nikinmaa:92, sievanen:93,  makela:97-1}
but,  instead of  aggregated tree  parts, it  has  a three-dimensional
description of the  above ground part of the  tree.  LIGNUM includes a
detailed    model    of    self-shading    within   a    tree    crown
\citep{perttunen:96,  perttunen:01}, from  which the  radiation regime
for photosynthesis in different parts of the tree can be computed.  If
the photosynthates produced exceed  the respiration costs then the net
production is allocated  to the growth of new  parts.  LIGNUM has been
applied to both coniferous \citep{perttunen:96,lo:99} and broad-leaved
trees \citep{perttunen:01}.  The main focus  in LIGNUM has been on the
functional  part of  the  model and  less  emphasis has  been paid  to
structural development.  The model  does not include any formal method
to define the architectural development of the tree structure.

We describe in the following how  the LIGNUM model has been interfaced
with L systems for  specifying formally the architectural  development
of trees, thereby improving the applicability and ease  of use of this
FSTM.   The goal  is achieved  by using  the L  language, which is  an
extension of L systems \citep{pp:99a}.  Based  on the definition of L,
R.  Karwowski created the original parser of L and has implemented the
L+C   language   \citep{karwowski:02}.    He   also  included  further
improvements not present in L, most  notably productions with multiple
successors and  the concept of   new context  \citep{karwowski:03}  to
allow fast linear time information transfer in the simulated plant.

We first present the  use of L in LIGNUM based on  the likeness of how
LIGNUM and bracketed L systems represent branching structures of trees
\citep{perttunen:96, perttunen:01}.  Second, we implement an algorithm
that translates  the bracketed  string of symbols  in L to  the LIGNUM
representation of trees (Section \ref{sec:pine}).  Third, we provide a
communication  mechanism  between the  L  string  and LIGNUM  (Section
\ref{sec:LToLignum}), and give two example applications, one for Scots
pine  and the  other for  bearberry  (\textit{Arctostaphylos uva-ursi}
L.).  Finally,  we discuss  the consequences of  our approach  for the
future development of LIGNUM in modelling trees and forest stands.

