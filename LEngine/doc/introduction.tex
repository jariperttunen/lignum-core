\section{Introduction}

Functional-structural tree models (FSTM) try to determine dynamics and
growth of woody perennial  plants by assessing physiological processes
in  their  (three dimensional)  arborescent  form.  The  physiological
processes or  plant metabolism involves for  example photosynthesis of
the foliage  depending on the  radiation regime encircling  the plant,
respiration, flow  of water, nutrients  or hormones and  allocation of
nutritive substances.   The structure of  the tree and its  changes in
time  is  described  for  example with  state  variables  representing
different  aggregated  tree compartments  or  with detailed  modelling
components faithful to botanical or morphological units of plants.

The term 'functional-structural model' was introduced in [M�kel� et al
97] that compiles a series of articles where the focus was set on what
happens in  a single  plant element (such  as shoot,  internode, leaf,
root hairs etc.)  and  defining their interactions.  Obviously, due to
possibly thousands of interacting units such models are implemented as
computer programs where the program acts as the model and as the means
to follow,  solve and analyse the model  behaviour (\textit{In silico}
simulations).

The research group at the  Department of Forest Ecology, University of
Helsinki  ('Helsinki school')  has  since early  1970's monitored  and
modelled  functioning  of  trees  and  their  environment  [Hari  99].
Analysis  of the  collected data  originally focused  on understanding
metabolism of  trees but the work  has extended to  analyze growth and
development of individual trees and forest stands.

The precursory modelling work [Hari  et al 82] for Scots pine presents
the framework for the stand  models based on process-based single tree
models where the  growth of trees is computed  from photosynthesis and
respiration.  The light regime in  a forest stand is the driving force
of  photosynthesis  in  a  tree  from  which  dry  matter  production,
maintenance  and allocation to  different tree  compartments (foliage,
stem, branches, root system  etc.)  represented as state variables are
contrived using the principle of carbon balance.

The  further analysis  of  the modelling  approach  has improved  tree
models  and  lead to  deeper  understanding  the  mechanics of  carbon
allocation,  structural  constraints controlling  tree  form and  tree
interactions supporting the further development of stand models.

To  mention a few,  [M�kel� and  Hari 86]  describe tree  growth using
photosynthetic  light   ratio  to  measure   competitive  ability  and
differentiation of  trees.  [Nikinmaa  92] has optimization  model for
photosynthesis  for  single  tree  involving light  extinction  within
canopy and construction cost  branches; stand level dynamics considers
nutrient   flows  explicitely.    [Siev�nen  93]   studies  structural
relationships  i.e.  pipe  model  for stem,  branches  and foliage  to
express   tree  and   stand   growth  in   terms  of   height-diameter
relationship.   Tree  and  stand   model  in  [M�kel�  97]  represents
geometric dimensions of woody organs  of stem, branches and roots with
constant form parameters.  [M�kel� et al 97] integrates two additional
submodels  to calculate  these parameters  explicitely.  Result  is an
explicit  representation of a  three dimensional  structure of  a tree
stem used in wood quality (branchiness) analysis of Scots pine.

LIGNUM is  a functional-structural single tree model  [Perttunen et al
96] faithful to modelling  approach described (see e.g.  [Nikinmaa 92,
Siev�nen 93] and [M�kel� 97])  but instead of aggregated tree parts it
has  three dimensional  description of  the above  ground part  of the
tree. Modeling the  functioning of the tree follows  the principles in
[Hari et  al 82].   LIGNUM includes a  detailed model  of self-shading
within a  tree crown [Perttunen et  al 98, Perttunen et  al 2001] from
which the  radiation regime for  photosynthesis in different  parts of
the  tree can  be  computed.  If  the  photosynthates produced  exceed
respiration costs the net production  is allocated to new and existing
tree structure  acquiring carbon balance. LIGNUM has  been applied for
both coniferous [Perttunen  et al 96, Lo et al  2000] and broad leaved
trees [Perttunen et al 2001].

The main  focus in LIGNUM has  been to develop functional  part of the
model.  Until now there has not been any method to formally define the
architecural development of the tree structure.  Research has produced
remarkable results in  this field however.  A leaf  of Barnsley's fern
is a prominent example of  fractal theory [Mandelbrott 77] providing a
way   to   describe   the   self  similarity   of   natural   objects.
Lindenmayer-systems (L-systems)  invented 1968 by  Aristid Lindenmayer
[Lindenmayer 68,71]  were initially meant to  describe the development
of multicellular organisms. L-systems are string rewriting systems and
their  research is  concerned  what phenomena  can  be described  with
formal  languages.   Prusinkewicz,  Kurth  and other  scientists  have
developed  both the theory,  tools and  applications for  plants using
L-systems framework. [Halle et  al 1970] have studied tropical botany,
analysed tree structure and  produced 25 architectural models based on
branching  pattern,  senescense and  functioning  of  buds.  Based  on
Halle's models  de Reffye [overview  de Reffye et  al 88, 90,  97] has
developed botanically  firm automaton theory  and the theory  has been
put into practice in AMAP software.
 



Schools:

-AMAP 'French School' (structure-->function)

-'L-System School' (language)

-'Helsinki School'  (function-->structure)


extending the  traditional three  step analysis of  a plant  where the
nature  of the  intrinsic processes  of the  entity are  analysed, the
differential  equations  describing  the  processes are  inferred  and
solved possibly  assuming some reasonably sufficient  geometry for the
entity.

By  looking  The  two  perspectives  of  plant  modelling,  structural
dynamics  and  physiological  processes,  have  evolved  independently
simplifying  or  simply  neglecting   the  standpoint  of  the  other.
However, the progress  in these two lines of  modelling have gradually
involved characteristics of each other bringing them closer.



LIGNUM is an FSM applied to 
Current interest

Objectives of this study.
 
Winfried  Kurth  has   classified  existing  modelling  approaches  in
forestry  into   three  categories  and  based   on  this  observation
constructed a model triangle.  The  top apex of this triangle is taken
up by aggregated  stand models.  Process based single  tree models and
architectural tree models  are placed at the base  apices of the model
triangle.  Moving along the sides and inside the triangle captures the
model continuum including aspects from  one or the other two modelling
avenues. 

In  Montpellier, a  city with  over  10 centuries  of known  botanical
heritage, the modelling group AMAP  at CIRAD ('French school') has the
basis for  their plant growth  and analysis systems  AMAPsim, AMAPpara
and AMAPmod  in the architectural analysis by  Halle\'{e} grouping the
morphological caharacteristics of tropical trees into 25 model classes
and the notion  of reiteration observed by Oldeman  where a plant body
is able to duplicate its existing structure.

AMAP re-introduced  the notion of  physiological age as  distinct from
real age  to organize the irreversible stochastic  processes of birth,
growth, branching and death of an apical bud (meristem). Each stage of
the bud  from birth to  death can be  arranged into a  theoretical one
dimensional reference  axis (\textit {axe  de r\'{e}f\'{e}rence}) with
shifts  from one  stage to  another as  possible paths  to  modify the
physiological age of  the bud.  The associated finite  automaton has a
state for each stage in  the reference axis, a transition function for
the shifts between the stages and additional geometric information for
branching and  growth. This finite  automaton system is  sufficient to
determine    architectural   tree   growth    including   reiteration,
self-pruning   and   changes  in   form   as   tree  develops   called
metamorphosis.  The  reference axis  was fully implemented  in AMAPsim
used  to produce  realistic images  of important  trees and  plants in
forestry and agriculture.
