\section{Introduction}
An increasing number of models  (see e.g. Annals of Forest Science vol
57 no.  5/6) try to determine  dynamics and growth  of woody perennial
plants by assessing physiological processes in their three dimensional
arborescent  form.    Physiological  processes  involve   for  example
photosynthesis of  the foliage, respiration, flow  of water, nutrients
or hormones and allocation  of nutritive substances.  The structure of
the tree and  its changes in time is described  for example with state
variables representing different  aggregated tree compartments or with
detailed modelling  components faithful to  botanical or morphological
units of  plants. Such  models have been  called functional-structural
(FSTM).

W.  Kurth has  proposed a  model triangle  \citep{kurth:94b} arranging
existing modelling approaches in  forestry into three categories based
on the  emphasis in the modeling approaches  (structure vs.  function,
single tree vs. forest  stand).  Accordingly there are practically two
ways to construct an FSTM  \citep{sievanen:00}.  One can start from an
architectural  model   \citep{jaeger:92,kurth:94}  adding  functional,
physiological details into it.  The second approach is to begin with a
process  based  physiological  model  \citep{makela:86,  landsberg:86,
sievanen:93} and extend it with structural details.

Linking these  two kinds of  models together rises the  question about
the suitable way of  expressing them. Physiological models usually are
realized  with  the  aid   of  differential  or  difference  equations
\citep{landsberg:86}  whereas architectural  models  apply Lindenmayer
systems  \citep{kurth:99,pp:90}  or  other  formalisms. Would  a  FSTM
combine those means or rely on either of them?

The model LIGNUM  approaches FSTM from the physiological  side; it is a
single  tree  model  \citep{perttunen:96}  faithful  to  process-based
modelling  \citep[see  e.g.][]{nikinmaa:92, sievanen:93,  makela:97-1}
but  instead  of  aggregated  tree  parts  it  has  three  dimensional
description of the  above ground part of the  tree.  LIGNUM includes a
detailed    model    of    self-shading    within   a    tree    crown
\citep{perttunen:96, perttunen:01} from which the radiation regime for
photosynthesis in different parts of the tree can be computed.  If the
photosynthates produced exceed respiration costs the net production is
allocated to the growth of new  parts. LIGNUM has been applied to both
coniferous   \citep{perttunen:96,lo:99}   and   broad   leaved   trees
\citep{perttunen:01}. The main focus in LIGNUM has been on the
functional  part  of  the  model;  less  emphasis  has  been  paid  to
structural  development, and  the model  does not  include  any formal
method to define the architectural development of the tree structure.

Since the pioneering work  by \citet{honda:71} there exist methods for
treating   plant  architecture   in  models   in  an   efficient  way.
Lindenmayer-systems or L-systems for  short \citep{pp:89} are the most
widely  used  method  to   treat  plant  architecture  although  other
formalisms   also  exist \citep[e.g.][]{dereffye:97,   godin:99}.
L-systems   were  invented   in   by  Aristid   \citet{lindenmayer:68,
  lindenmayer:71} and were initially meant to describe the development
of  multicellular organisms.  L-systems  are string  rewriting systems
and their research  is concerned with the question  what phenomena can
be   described  with   formal  languages.    The  theory,   tools  and
applications using  L-systems framework for modeling  plants and their
environment have been  developed by Prusinkiewicz \citep{pp:89,pp:92},
Kurth  \citep{kurth:94} and other  scientists.  The  ubiquitous theory
and the progress in L-systems  in modelling plants and trees until the
end   of  '90's   is  well   documented  in   \citet{pp:90,pp:99}  and
\citet{kurth:99}.

We describe in the following  how the model LIGNUM has been interfaced
with   L-systems   for   specifying   formally  and   rigorously   the
architectural   development  of  trees   and  thereby   improving  the
applicability and  ease of using this  FSTM.  The goal  is achieved by
using the language L that  is an extension of L-systems \citep{pp:99a}.
Based on the definition of  L, R.  Karwowski has created the original
parser of L and  has implemented the language L+C \citep{karwowski:02}
including  further  improvements  not   present  in  L,  most  notably
productions with multiple successors and the concept of new context
\citep{karwowski:03} to allow fast linear time information transfer in
the simulated plant.

We first  present the use of L  in LIGNUM based on  the similarity how
LIGNUM and bracketed L-systems represent branching structures of trees
\citep{perttunen:96, perttunen:01}.  Second, we implement an algorithm
that  translates  the bracketed  string  of  symbols  in L  to  LIGNUM
representation of  trees (Section \ref{sec:pine}).   Third, we provide
communication  mechanism  between  the  L-string and  LIGNUM  (Section
\ref{sec:LtoLignum}) and give two  sample applications, Scots pine and
bearberry.   We present simple  language construct  that allows  us to
model   group  of  plants   each  with   its  own   L-system  (Section
\ref{sec:namespace}).   Finally,  we  discuss  our  approach  for  its
consequences to the future development of LIGNUM in modeling trees and
forest stands.

