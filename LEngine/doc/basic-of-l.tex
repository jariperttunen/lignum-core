\section{Basics of L-systems}

Mathematically L-systems  are parallel rewriting  systems operating on
strings of symbols. An L-system  is defined by an alphabet of symbols,
a  set of  rewriting rules  called productions  and an  initial string
called axiom. In a production  written as $a \rightarrow S$ the symbol
$a$  is   called  predecessor  and  the  string   $S$  successor.   In
context-free  L-systems  predecessor's   context  does  not  influence
production  application.  In  context-sensitive  L-systems written  as
$S_l < a  > S_r \rightarrow S$, the symbol $a$  can produce string $S$
if and  only if  $a$ is  preceded by string  $S_l$ (left  context) and
followed  by string  $S_r$ (right  context).  For  example  define the
alphabet of three  symbols $O$, $X$ and $S$ and  set of following nine
rules:
\begin{equation}\label{eq:ca110}
\begin{array}{lll}
1:O < O > O \rightarrow O & 2:O < O > X \rightarrow X &
3:O < X > O \rightarrow X \\ 
4:O < X > X \rightarrow X & 5:X < O > O \rightarrow O&
6:X < O > X \rightarrow X \\
7:X < X > O \rightarrow X & 8:X < X > X \rightarrow O & 
9:S > O \rightarrow SO
\end{array}
\end{equation}
enlisting the $2^3$  possible ways of $O$ and $X$  to interact and the
ninth rule expanding the string.  Starting from the axiom $SOOOXO$ the
first four  strings produced by the L-system  are $SOOOXO$, $SOOOXXO$,
\linebreak $SOOOXXXO$ and $SOOOXXOXO$.

In plant modeling  context the symbols of the  alphabet in an L-system
represent the units (internodes, leaves, flowers etc.)  of the growing
organism relevant to modeling approach and the string of symbols their
topological ordering.  To be able to generate the branching structures
dominant  in  plant world  the  notion  of  strings with  brackets  or
bracketed  L-systems was  already  present in  the original  formalism
\citep{lindenmayer:68}. To describe the geometry and for the graphical
interpretation  of  the   model  structures  \citet{pp:86}  introduced
L-system  symbols  as  instructions  controlling a  LOGO-style  turtle
\citep{abelson:82} in three dimensions,  hence the adopted name Turtle
graphics.

