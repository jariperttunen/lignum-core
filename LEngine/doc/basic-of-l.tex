\section{Principles of L-systems}

Mathematically L  systems are parallel rewriting  systems operating on
strings of symbols. An L system  is defined by an alphabet of symbols,
a set  of rewriting  rules called productions,  and an  initial string
called an  axiom. In  a production written  as $a \rightarrow  S$, the
symbol $a$ is called the predecessor and the string $S$ the successor.
For example, define  the alphabet of two symbols $A$  and $B$, and the
set of following two rules:

\begin{equation}\label{eq:AB}
1: A \rightarrow B \qquad 2: B \rightarrow AB 
\end{equation}
Starting from the axiom $A$, the  first five strings produced by the L
system are $A$, $B$, $AB$ and $BAB$ and $ABBAB$.

In context-free  L systems (as in Eq.   \ref{eq:AB}) the predecessor's
context   does   not  influence   the   production  application.    In
context-sensitive L systems, written as $S_l < a > S_r \rightarrow S$,
the symbol $a$ can produce a string $S$ if and only if $a$ is preceded
by the  string $S_l$ (left context)  and followed by  the string $S_r$
(right context).

In the  plant modeling  context the  symbols of the  alphabet in  an L
system represent the units  (internodes, leaves, flowers etc.)  of the
growing organism relevant to the  modeling approach, and the string of
symbols  their topological  ordering.   To be  able  to generate,  the
branching structures dominant in the plant world the notion of strings
with brackets  (literally denoting the  beginning and end of  a branch
with '['  and ']') or bracketed  L systems was already  present in the
original  formalism \citep{lindenmayer:68}.   In parametric  L systems
\citep{pp:90a},  symbols  may  take  numerical  arguments  to  capture
conveniently continuous phenomena like  the diffusion of substances in
an organism.  Symbols in parametric L systems are also called modules.
To describe the geometry, and also for the graphical interpretation of
the  model structures,  \citet{pp:86} introduced  L-system  symbols as
instructions  controlling a  LOGO-style  turtle \citep{abelson:82}  in
three dimensions, hence the adopted name Turtle graphics.

