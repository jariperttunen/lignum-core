\section{Introduction to the L language}

Mathematically L systems\footnote{We present all examples of L systems
  using language L notation \citep{karwowski:02} instead of the
  commonly used notation \citep{pp:89}.} are parallel rewriting
systems operating on strings, i.e.  sequences of symbols. An L system
is defined by an \it alphabet \rm of symbols, a set of rewriting rules
called \it productions\rm, and an initial string called an \it
axiom\rm.

In the  plant modelling context  the symbols of  the alphabet in  an L
system represent the units  (internodes, leaves, flowers etc.)  of the
growing organism relevant to the modelling approach, and the string of
symbols  their  topological ordering.   To  be  able  to generate  the
branching  structures  dominant in  the  plant  world,  the notion  of
strings with brackets  (literally denoting the beginning and  end of a
branch with '[' and ']') or bracketed L systems was already present in
the  original  formalism   \citep{lindenmayer:68}.   

In  parametric  L systems \citep{pp:90a},  symbols  may take numerical
arguments to depict properties of the units like size or cocentrations
of   substances.   Symbols in  parametric   L systems  are also called
modules.  In context-free   L systems the predecessor's context,  i.e. 
modules on its left and right side, does  not influence the production
application.  In    context-sensitive  L  systems   the  preceding and
following modules affect the rewriting.

The key concept in the L  system formalism is the rewriting of modules
or symbols  \citep{pp:89}.  The  L language follows  the same  idea. A
module in  L has a name  and can take  any number of arguments  of any
type   in   C++   programming   language   \citep{stroustrup:97}.    A
syntactically  correct rule  in L  consists of  a  predecessor module,
possibly with its context ending with colon, and a production defining
successor  string embraced  within  curly braces.   A special  module,
$Start$, corresponds to the axiom. For example, define the alphabet of
two modules $A()$ and $B()$, the  axiom $A()$ and the set of following
two rules:

\begin{eqnarray}\label{eq:AB}  
\lefteqn{\mathrm{Start}:\{A();\} \nonumber }\\
\lefteqn{A():\{\mathrm{produce}\ B();\}   }\\
\lefteqn{B():\{\mathrm{produce}\ A()B();\nonumber \} }  
\end{eqnarray}

Starting from the axiom $A()$,  the first five strings produced by the
L system  in Eq. \ref{eq:AB}  are $A()$, $B()$,  $A()B()$, $B()A()B()$
and $A()B()B()A()B()$. Note the parallel rewriting of modules.


To describe the geometry, and also for the graphical interpretation of
the  model structures,  \citet{pp:86} introduced  L system  symbols as
instructions  controlling a  LOGO-style  turtle \citep{abelson:82}  in
three dimensions, hence the adopted name turtle graphics.

For branching structures of plants L predefines two modules $SB()$ and
$EB()$ denoting  the beginning and  end of a branch  respectively.  To
control the movements of the turtle (the geometry engine) let us first
define its orientation in space  by three unit vectors $\vec H$, $\vec
L$ and $\vec  U$ denoting the turtle's heading,  direction to the left
and up at right angles to each other such that $\vec U = \vec H \times
\vec L$.  Then  define module $F(s)$ so as to  move the turtle forward
along   its  heading   step   of  length   $s$,   and  three   modules
$Turn(\alpha)$,   $Pitch(\alpha)$,   $Roll(\alpha)$   to  rotate   the
orientation of  the turtle around $\vec  U$, $\vec L$ and  $\vec H$ by
$\alpha$.  


