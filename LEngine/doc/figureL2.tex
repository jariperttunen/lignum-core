\section{Fragmentary  L system   for  bearberry  growth}\label{sec:L2}

Fragmentary L system for  bearberry growth. The $Start$ module creates
the initial  plant.  The  $B$ module, line  15, checks  for collision.
Lines  18--40   determine  branching  and  growth.    The  pattern  of
ramification is  based on  field data and  implemented in  the uniform
random variables $r1~\mathrm{and}~r2  \in [0,1]$.  $r1$ intializes the
branching to the  left or right.  Branching and  growth depends on the
bud type,  its status and the value  of $r2$.  The counter  on line 41
eventually activates dormant buds ($s > 0$).  Bud types: D = dominant,
N  = nondominant  and  S =  subdominant.   See \citet{salemaa:02}  for
details.

%\begin{figure}[p]
%\begin{picture}(1,1)
%\put(0,0){\line(1,0){370}}
%\end{picture}
\begin{verbatim}
 1.open Bearberry;
 2.const PI = 3.1415926535897932384;
 3.module B(double type,double status,double collision); 
 4.derivation length: 15;
 5.Start:{produce F(0.1) SB() EB() B(D,0.0,0.0);}
 6.B(T,s,C):
 7.{
 8.  double g = 0.0;
 9.  double r1 = ran();
10.  double r2 = ran();
11.  if (r1 < 0.5) 
12.     g = -5*PI/180;
13.  else 
14.     g =  5*PI/180;
15.  if (C == 1.0){
16.    produce B(T,s,C);
17.  }
18.  else if (T == D && s == 0.0){
19.    if (r2 < 0.26)
20.      produce Turn(g)F(0.6)SB()Turn( 30*PI/180)B(N,2,C)EB() 
21.              Turn(g)F(0.1)SB()Turn(-30*PI/180)B(N,1,C)EB()
22.              Turn(g)F(0.1)SB()Turn( 30*PI/180)B(S,1,C)EB()
23.              Turn(g)F(0.1)SB()Turn(-30*PI/180)B(S,0,C)EB()
24.              Turn(g)F(0.1)SB()EB()B(D,0,C);
25.    else if (r2 <= 0.52)
26.      produce Turn(g)F(0.6)SB()Turn(-30*PI/180)B(N,2,C)EB() 
27.              Turn(g)F(0.1)SB()Turn( 30*PI/180)B(N,1,C)EB()
26.              Turn(g)F(0.1)SB()Turn(-30*PI/180)B(S,1,C)EB()
28.              Turn(g)F(0.1)SB()Turn( 30*PI/180)B(S,0,C)EB()
29.              Turn(g)F(0.1)SB()EB()B(D,0,C);
30       ......................................................
32.  } 
33.  else if (T == S && s == 0.0){
33.    if (r2 < 0.037)
34.      produce Turn(g)F(0.48)SB()Turn(-30*PI/180)B(S,1,C)EB()
35.              Turn(g)F(0.08)SB()EB()B(S,0,C);
36.       ......................................................
37.  }
38.  else if (T == N && s == 0.0){
39.       ......................................................
40.  else{
41.    produce B(T,max(s-1,0),C);
42.  }
43.}
44.close Bearberry;
\end{verbatim}
%\begin{picture}(1,1)
%\put(0,0){\line(1,0){370}}
%\end{picture}
%\caption{)
%\end{figure}

