\begin{figure}[p]
\begin{picture}(1,1)
\put(0,0){\line(1,0){370}}
\end{picture}
\begin{verbatim}
 1.open Bearberry;
 2.module B(double type,double status,double collision); 
 3.derivation length: 10;
 4.Start:{produce F(0.1) SB() EB() B(D,0.0,0.0);}
 5.B(T,s,C):
 6.{
 7.  double g = 0.0;
 8.  double r1 = ran();
 9.  double r2 = ran();
10.  if (r1 < 0.5) 
11.     g = -5;
12.  else 
13.     g = 5;
14.  if (C == 1.0){
15.    produce B(T,s,C);
16.  }
17.  else if (T == D && S == 0.0){
18.    if (r2 < 0.0875)
19.      produce Turn(g) F(0.70) SB() Turn(52)  B(N,2,C) EB() 
20.              Turn(g) F(0.15) SB() Turn(-30) B(D,0,C) EB()
21.              Turn(g) F(0.15) SB() EB() B(D,0,C);
22.    ......................................................
23.    else if (r2 <= 1.0)
24.      produce Turn(g) F(0.50) SB() Turn(-52) B(N,2,C) EB() 
25.              Turn(g) F(0.10) SB() Turn(52)  B(N,1,C) EB()
26.              Turn(g) F(0.10) SB() Turn(-31) B(S,1,C) EB()
27.              Turn(g) F(0.10) SB() Turn(31)  B(S,0,C) EB()
28.              Turn(g) F(0.20) SB() EB() B(T,0,C);
29.  } 
30.  else if (T == S && S == 0.0){
31.    ......................................................
32.  }
33.  else{
34.   double s = maximum(s-1.0,0.0);
35.   produce B(T,s,C);
36.  }
37.}
\end{verbatim}
\begin{picture}(1,1)
\put(0,0){\line(1,0){370}}
\end{picture}
\caption{Fragmentary  L-system for bearberry growth. Module Start creates 
  the initial  plant. In the module B  line 14 check  for collision, lines
  17--32 determine  branching and growth. The  pattern of ramification
  is  based  on field  data  and  implemented  in the  uniform  random
  variables  $r1~\mathrm{and}~r2  \in  [0,1]$.   $r1$  intializes  the
  branching to  the left  or right.  Growth  depends on bud  type, its
  status and  the value  of $r2$.  The  counter on line  34 eventually
  activate dormant  buds ($s > 0$).  For  bud types D =  dominant, \mbox{N =
  nondominant} and S = subdominant.  See \citet{salemaa:02} for details
  .}\label{fig:L2}
\end{figure}

