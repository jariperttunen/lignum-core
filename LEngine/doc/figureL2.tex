\begin{figure}[p]
\begin{picture}(1,1)
\put(0,0){\line(1,0){370}}
\end{picture}
\begin{verbatim}

open Bearberry;

#include <lsys.h>


const int D    = 1.0; //dominant    apical
const int S    = 2.0; //subdominant apical
const int T    = 3.0; //subdominant lateral
const int N    = 4.0; //nondominant apical
const int O    = 5.0; //nondominat  lateral

const double DL = 1;    
const double SL = 0.58;  
const double NL = 0.40;  
const double DA = 30*2.0*PI_VALUE/360.0;
const double SA = 31*2.0*PI_VALUE/360.0;
const double NA = 52*2.0*PI_VALUE/360.0;
const double CU = -5.0*2.0*PI_VALUE/360.0;
const double CT = 0.5*2.0*PI_VALUE/360.0;

module B(double,double,double); 

derivation length: 10;
int seed

Start:
{
  int init = -1;
  int seed = 137591;
  ran3(&init);
  produce F(0.1*DL) SB() EB() B(D,0.0,0.0);
}

B(T,S,C):
{
  double g = 0.0;
  double r1 = ran3(&seed);
  double r2 = ran3(&seed);
  
  if (r1 < CT)
     g = CU;
  else 
     g = -CU;

  if (C == 1.0){
    produce B(T,S,C);
  }
  else if (T == D && S == 0.0){
    if (r2 < 0.0875)
      produce Turn(g) F(0.70*DL) SB() Turn(NA)  B(N,2,C) EB() 
              Turn(g) F(0.15*DL) SB() Turn(-DA) B(D,0,C) EB()
              Turn(g) F(0.15*DL) SB() EB() B(D,0,C);
    ............................................................
    else if (r2 <= 1.0)
      produce Turn(g) F(0.50*DL) SB() Turn(-NA) B(N,2,C) EB() 
              Turn(g) F(0.10*DL) SB() Turn(NA)  B(N,1,C) EB()
              Turn(g) F(0.10*DL) SB() Turn(-SA) B(S,1,C) EB()
              Turn(g) F(0.10*DL) SB() Turn(SA)  B(S,0,C) EB()
              Turn(g) F(0.20*DL) SB() EB() B(T,0,C);
  } 
  else if (T == S && S == 0.0){
    if (r2 < 0.257)
     produce Turn(g) F(0.65*SL) SB() Turn(NA)  B(N,2,C) EB()
             Turn(g) F(0.15*DL) SB() Turn(-SA) B(S,1,C) EB()
             Turn(g) F(0.20)    SB() EB() B(S,0,C);
    ............................................................
    else if (r2 <= 1.0)
     produce Turn(g) F(0.65*SL) SB() Turn(-NA)  B(N,2,C) EB()
             Turn(g) F(0.15*DL) SB() Turn(SA) B(S,1,C) EB()
             Turn(g) F(0.20)    SB() EB() B(T,0,C);   
  }  
  else if (T == N && S == 0.0){
    if (r2 < 0.630)
     produce Turn(g) F(NL) SB() EB() B(N,0,C);
    ............................................................
    else if (r2 <= 1.0)
     produce Turn(g) F(0.75*NL) SB() Turn(-NA) B(N,1,C) EB()
             Turn(g) F(0.25)    SB() EB() B(T,0,C);
  }
  else{
   double s = maximum(S-1.0,0.0);
   produce B(T,s,C);
  }

}
\end{verbatim}
\begin{picture}(1,1)
\put(0,0){\line(1,0){370}}
\end{picture}
%\hline
\caption{L-system for bearberry growth [Salemaa and Siev�nen 02]}\label{fig:L2}
\end{figure}