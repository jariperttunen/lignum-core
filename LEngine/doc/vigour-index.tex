\section{Vigour index}\label{sec:vi}
[Nikinmaa ??] and [Goulet et al  00] have studied the effect of branch
position on branch  segment growth with the help  of vigor index.  The
value of vigor index,$V$, is based on the relative thicknesses of axes
forking away from a branching point. The vigor index was formulated in
[Perttunen et al 01] for sugar maple sapling as follows:
\begin{eqnarray}
  \label{eq:vi}
  V_{\mathrm{base}} & = & 1 \nonumber \\
  V_{\mathrm{n}} & = & \left(\frac{A_{\mathrm{ts_{n}}}}{\max\left[A_{\mathrm{ts_{i}}}\right]}\right) V_{\mathrm{below}}
\end{eqnarray}

$A_{\mathrm{ts_{n}}}$ is  the cross sectional area of  segment $n$ and
the  maximum  of  $A_{\mathrm{ts_{i}}}$  is taken  over  all  segments
forking    from   the    branching   point    towards    the   apexes.
$V_\mathrm{below}$ is  the vigor index of segment  below the branching
point.  The  recursion starts from  the first segment  $V_{base}$ with
value 1. The thicker the axis the more growth potency it has.

To  express the  Eq.  \ref{eq:vi}  with L-system  rule first  note the
computation  of the  vigor index  is sequential  but the  rewriting in
L-systems is parallel. Thus one can  not simply write for example $F >
\left[F\right] \left[F\right]  F$ and  the production  to implement
Eq. \ref{eq:vi}.

To write  rules to  compute vigor index  first redefine module  $F$ as
$F(s,A,v)$ where  $A$ is the  cross-sectional area of the  segment and
$v$  its  vigor index.   Then  define  module  $V$ to  syncronize  the
computation sequentially in a tree.  The two L-system rules needed are
in Fig. \ref{fig:vi}.  The first rule moves module  $V$ upwards in the
tree  and  the  second rule  removes  it  from  the string  after  the
computation is done.


