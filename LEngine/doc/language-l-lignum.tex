\section{Introduction to language L}

The key concept in the  L-system formalism is the rewriting of modules
or symbols \citep{pp:89}.  The language  L follows the same concept. A
module in  L has a name  and can take  any number of arguments  of any
type   in   C++   programming   language   \citep{stroustrup:97}.    A
syntactically correct rule consists of a predecessor possibly with its
context ending  with colon and a production  defining successor string
embraced within curly braces.  For example the third rule in Eq.
\ref{eq:ca110} is written in L as:
\begin{equation}\label{eq:r3}
 O() < X() > O(): \{\mathrm{produce}\ X();\}
\end{equation}

For branching structures of plants L predefines two modules $SB()$ and
$EB()$ denoting begin and end  of branch respectively.  To control the
movements of the turtle (the  geometry engine) let us first define its
orientation  in space by  three unit  vectors $\vec  H$, $\vec  L$ and
$\vec  U$ denoting  turtle's heading,  direction  to the  left and  up
perpendicular to each other such that $\vec U = \vec H \times \vec L$.
Then define module $F(s)$ to move the turtle forward along its heading
step of length $s$  and three modules $Turn(\alpha)$, $Pitch(\alpha)$,
$Roll(\alpha)$ to  rotate orientation of  the turtle around  $\vec U$,
$\vec  L$  and $\vec  H$  by  $\alpha$.   One special  module  $Start$
corresponds to the axiom.


\section{Integrating the language L and LIGNUM}\label{sec:pine}

To model tree structures with L as in LIGNUM let the module $F$ denote
tree segment and designate module  $B$ for bud.  Given the modules for
turtle  graphics  and  branching  we  can now  describe  topology  and
geometry of  the tree structures  of LIGNUM (Eq.   \ref{eq:tree}) with
rules of  the language  L (Eq.  \ref{eq:r3}).   To accommodate  L with
LIGNUM we need first to construct an algorithm that translates strings
of L to structural units of LIGNUM (Section \ref{sec:LToLignum}).  Second,
as  metabolism implemented  in  LIGNUM will  act  on structural  units
(segments  and  buds  most  notably)  and produces  changes  in  their
physiological states,  in their dimensions  and in their  positions in
space  accordingly, this  information  must be  transferred to  turtle
symbols  and module  $B$  as  it is  necessary  in rewriting  (Section
\ref{sec:LignumToL}).

We present  an algorithm  that translates strings  of L  to structural
units  of LIGNUM  with  the aid  an  example on  Scots  pine. We  have
programmed in  language L the architectural development  of Scots pine
as it  is approximately in LIGNUM  simulations of \citet{perttunen:96,
  perttunen:98}; the L program  is shown in Appendix \ref{sec:L1}.  In
the program  the module $B(A,L)$  represents a bud. In  each iteration
the apical  bud ($A =  1$) moves forward  of length $L$ and  forks off
four new side  branches.  The branches ($A > 1$)  are similar but only
two more side branches are created.  Branching stops after third order
($A > 3$).   When the development starts from  main axis consisting of
one segment, branching point and  bud, the string after two iterations
is

\begin{equation}\label{eq:pine2}
F\;[] F [F[B][B]B]\:[F[B][B]B]\:[F[B][B]B]\:[F[B][B]B]\; F \;[B][B][B][B]\; B
\end{equation}

where symbol [ denotes $SB()$ and ] denotes $EB()$, modules for turtle
rotations are not shown, and the  arguments of the modules $B$ and $F$
are dropped.

\subsection{From L to LIGNUM}\label{sec:LToLignum}

We can now  outline a straightforward algorithm to  convert the string
of symbols in L to structural units in LIGNUM using Eq. \ref{eq:pine2}
as a particular example.  The  algorithm is based on first recognizing
the main axis  of the tree, then finding the  other axes (branches and
sub-brances), and finally grouping them together as branching points.

First, the symbol $F(s)$ in Eq.  \ref{eq:pine2} is interpreted as tree
segment  of length  $s$.  The  symbol  $B$ corresponds  to bud.   Each
consecutive  set of $n$  branches between  two $F$  symbols will  be a
branching  point with  $n$ axes.   The string  in  Eq.  \ref{eq:pine2}
becomes  first the  main  axis with  three  segments, three  branching
points and the terminating bud:

\begin{equation}
[TS, BP, TS, BP, TS, BP, B]
\end{equation}

The  two  last branching  points  contain  four  axes each,  the  first
branching point being empty:

\begin{equation}
[TS, [], TS, [A,A,A,A], TS, [A,A,A,A], B]
\end{equation}

Finally, recursively constructing the axes we get:
\begin{equation}
[TS, [] TS,[TS,[[B],[B]],B],\ldots, [TS,[[B],[B]],B]], TS, [[B],[B],[B],[B]], B]
\end{equation}

Current  status of  the turtle  that  is updated  according to  turtle
commands in  the string  defines the position  and orientation  of the
tree compartments in LIGNUM.

The structural development of the tree described in L does not require
the regeneration  of LIGNUM representation of trees  (deleting the old
tree and  creating a new  one) after each derivation.   The algorithms
assumes that an L-system indeed generates alternating sequence of tree
segments, branching points and  terminating buds and the terminal buds
only generate  new structural units.   The algorithm fails if  this is
not the case.  Thus after each  derivation it is possible to match the
existing  string   and  LIGNUM  representation  and   insert  the  new
structural units in  the axis lists before the  terminating buds whose
positions and orientations will be updated.


\subsection{From LIGNUM to L}\label{sec:LignumToL}

Because architectural  development of a tree is  defined with L-string
we can assume LIGNUM does not have to change the topology of the tree.
Consequently the conversion from LIGNUM tree to L-string is trivial.
  
But  as the  physiological activities  in  LIGNUM \citep{perttunen:96}
will change the dimensions and  statuses of tree segments and buds, it
is necessary  that information can  be transferred from LIGNUM  to the
interpretation   of  the   L program.    That  is,   results  of   the
physiological processes in LIGNUM must be able to change the parameter
values of  the modules  $F$ and $B$  in L string  enabling interaction
between the architectural part and the physiological part.

Since the parameter $s$ in  the module $F(s)$ is always interpreted as
the length the turtle moves forward and its meaning is the length of a
segment in  LIGNUM the conversion algorithm can  implicitly update the
value of $s$ using the length of the corresponding tree segment.

But the parameters for module $B$ are model specific.  Thus explicitly
given  the (C++)  type of  the parameters  for the  module  $B$, their
assignment statements and their  variable names (meaning) in LIGNUM to
the  conversion algorithm  updates the  module $B$'s  parameter values
using corresponding  variable values  from the associated  bud passing
the results  of computations in  LIGNUM to L-system.   (Similarly, the
values of  parameters of  module $B$ can  be assigned to  variables of
corresponding bud in LIGNUM in Section \ref{sec:LToLignum}).

\subsection{Two-way communication in Scots pine simulation}

In  our  Scots  pine example  although  it  is  $L$ in  $B(A,L)$  that
initially determine the length  of a segment the metabolical processes
eventually resolve the amount of growth.
  
We  interfaced the L  language program  of Appendix  \ref{sec:L1} with
functional part of LIGNUM that incorporates our previous work on Scots
pine  model \citep{perttunen:96,  perttunen:98}.  The  calculations in
LIGNUM  include the  pairwise comparison  of segments  to  compute the
interception of  solar radiation and  the iterative allocation  of net
photosynthates to growth after respiration costs: 
\begin{equation} 
P - M  = iW_n  +  iW_o +  iWr 
\end{equation}  

where  $P$   and  $M$  are  the  photosynthetic   production  and  the
respiration of a  tree, $iW_n$ the growth of  new segments, $iW_o$ the
secondary      growth     and      $iW_r$     the      root     growth
\citep[c.f.][]{perttunen:96,perttunen:01}.

The simulations  show that the  interaction of the L  language program
and  the  functioning  part  of  LIGNUM affects  the  outcome  of  the
simulation markedly  (Figure \ref{fig:pine}).  Also the effect  of two
extreme   cases   of  foliage   mortality   show  that   physiological
characteristics have a great effect  on the architecture of the trees.
The simulation  shows much shorter and  slimmer stem due  to less need
for  diameter  growth  for  the  pine with  short  living  foliage  in
comparison to pine with long living foliage.

