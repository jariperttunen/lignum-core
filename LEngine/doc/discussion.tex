\section{Discussion}

Trees are complex modular systems \citep{thomas:00}.  Their growth and
development  has  been studied  intensively  during  the last  century
mainly experimentally.   This has resulted statistical  models and the
emphasis has been  on the details of measurements,  the arrangement of
experiments  and  the  use  and  development  of  statistical  methods
\citep{hari:99}.   LIGNUM  is  an  attempt  to link  a  detailed  tree
architecture with the knowledge in tree physiology based on measurable
functions and activities in a  tree. The attraction with such approach
is that it  is possible to make ontological  arguments (i.e. reasoning
based on  the being  itself and its  existence) of the  development of
trees  \citep{perttunen:01}.   For example  a  modeling framework  for
different formulations  of flushing  of buds or  number of  leaves and
their  orientation based  on light  regime or  resource  allocation is
readily available in LIGNUM.

This study presents a formal way to model architectural development of
trees in LIGNUM using L-systems  with the language L.  Essentially the
use of L is based on the similarity how bracketed L-systems and LIGNUM
represent the  branching structure of  trees (Section \ref{sec:pine}).
Similar conversions  between modeling  frameworks and tools  have been
reported for example  by \citet{ferraro:02}and \citet{dzierzon:03}. In
fact already \citet{kurth:em94}  has reported convergence between tree
models produced  by AMAP and  the same models expressed  in L-systems.
Also, a large body of  mathematical forms known as fractals describing
plant structures have  been described with L-systems \citep{kurth:99}.
The  Eq.  \ref{eq:ca110}  is an  example of  one  dimensional cellular
automata  system,   the  rule  110   in  Wolfram's  \citep{wolfram:02}
classification (the first eight  rules implement the cellular automata
system  and  the  ninth  rule  expands  the  string).   Interestingly,
according to  \citet{wolfram:02} the rule 110 is  capable of universal
computation.

L-systems provide good scientific abstraction needed in plant modeling
(c.f.    \citet{regev:02}).   An   L-system   captures  the   relevant
properties of  the phenomena in  its set of symbols  highlighting only
essential characteristics  of the model.  It  is computable supporting
qualitative and quantitative reasoning of the model properties.  It is
extensible, new  symbols can capture additional features  of the model
if required  and it is  understandable, the formal notation  allows to
share  and  compare scientific  knowledge.   For  example  part of  an
L-system  model  in  itself  can  appear in  a  publication  as  model
description  unlike   models  implemented  with   general  programming
languages.

The  theoretical  advancements later  implemented  in  tools based  on
L-systems formalism  has been motivated  by the curiosity to  find out
what  phenomena  in  plant  modeling  can be  formally  described  and
simulated.   The range  of circumstances  where L-system  formalism is
applicable  is quite extensive  \citep{pp:99}.  In  fact, it  is quite
possible  to  implement  many  physiological  models  in  LIGNUM  with
L-systems.    It  would   be   trivial  to   reimplement  models   for
photosynthesis and  respiration.  With a little effort  one can surely
outline a rule set for  the iterative allocation of net photosynthates
to  growth   \citep{perttunen:96,  perttunen:98,  perttunen:01}.   The
simulation of plant communities  has been reported by \citet{kurth:99}
and with multiset  L-systems developed by \citet{lane:02}.  Inevitably
one has to ask why not reimplement LIGNUM and make L-systems the basis
of its future development?

Plants are not closed systems  but interaction with environment has an
important function  in their  development.  Such modeling  include for
example computation of light regime in plant community and competition
for   growth   space   in   clonal  plant.    Bearberry   in   Section
\ref{sec:bearberry} is  an example of  the latter using  algorithm for
collision detection.   To express such  phenomena, \citet{mech:97} and
\citet{mech:96}  have extended  L-system formalism  with communication
symbols  that can  pass parameter  values between  the plant  model in
L-system  and  a  separate  program simulating  for  example  relevant
characteristics of its  environment.  \citet{kurth:94} has implemented
a  set of  predefined functions  to return  environmental information,
global  sensitivity, to  the  L-system in  GROGRA simulation  program.
Once implemented  these separate programs or  predefined functions are
easily  used and reused  but note  the time  spent to  implement these
environmental models is the time spent outside the formalism with some
general purpose  programming language.  Also,  the full implementation
of  bearberry  (Appendix  \ref{sec:L2})  is  about 190  lines  long  so
L-systems can become cumbersome  programs too but usually shorter than
same models created with general purpose programming languages.
 
An essential feature of the  original L-systems theory is its discrete
character.   Symbols are  created, rewritten  and deleted  in distinct
events  captured  by  the  rules.   This limitation  was  assessed  in
differential L-systems \citep{hammel:96}  where modules or symbols are
created and disappear in discrete time steps but develop in continuous
fashion described  by differential equations. But there  is still much
discreteness that remains.

An  important   class  of  chemical   reactions  can  be   modeled  as
reaction-diffusion equations. Suppose a  sequence of tree segments and
two  nutrients $A$  and $B$  present  in the  segments denoting  their
concentrations with $a$  and $b$.  Assume that the  amount of nutrient
$A$ depends  on the quantities of  $A$ and $B$ already  in the segment
and similarly for $B$.  This is  the reaction part of the system. Then
assume  the  two nutrients  can  dissolve  between adjacent  segments.
Typically if in a segment the concentration of the nutrients is higher
than  its neighbors  then the  segments' concentration  will  even out
because of  flow of  the nutrients away  from the  peak concentration.
This  is the  diffusion part  of the  system.  

Formally  one  typically   writes  $\frac{\partial  a}{\partial  t}  =
R_a(a,b) + C_a\nabla^2a$ for $A$'s changes in time and $\frac{\partial
  b}{\partial t}  = R_b(a,b) + C_b\nabla^2b$ for  $B$.  The Laplacians
$\nabla^2 a$ and $\nabla^2 b$ \citep{thomas:92} in diffusion terms are
the measure of  the concentrations in one location.   If the Laplacian
is  negative there is  a flow  towards that  point and  positive value
implies local peak concentration and flow away from that point.

Thus one can imagine a continuous  flow of nutrients in every point in
segments  in a  continuous  time expressed  with partial  differential
equations and perhaps design L-system formalism around it but how does
one write  rules having  no explicit time  steps?  In  general partial
differential  equations  are considered  difficult  to  solve and  the
knowledge of the  problem domain is used to  find efficient solutions.
Reaction-diffusion model  in the context of  branching structures (and
LIGNUM) is in \citet{palovaara:03}. 

It is still an open question for us what parts of the modeling efforts
with LIGNUM will switch to  L-systems.  Some modeling efforts done can
be  easily   expressed  with  rewriting  but   other  like  ubiquitous
computation of light regime, creation of leaves and the positioning of
leaf  blades etc.   inevitably  requires  at least  some  part of  the
implementation  outside the  formalism.  In  any case  it would  be an
interesting exercise  in itself to study carefully  the convergence of
LIGNUM  and L-systems by  fully implementing  LIGNUM in  some L-system
tool.

