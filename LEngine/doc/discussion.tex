\section{Discussion}

This  study   presents  a  formal  way  to   model  the  architectural
development of  trees using  L systems through  the language L  in the
functional-structural  tree model LIGNUM  implemented using  a general
purpose programming  language.  Essentially the  use of L is  based on
the similarity  of how  bracketed L systems  and LIGNUM  represent the
branching structure  of trees.  Similar  conversions between modelling
frameworks   and   tools  have   been   reported,   for  example,   by
\citet{ferraro:02}   and   \citet{dzierzon:03}.    In  fact,   already
\citet{kurth:94b} reported  convergence between tree  models produced
by AMAP  and the same  models expressed in  L systems.  Also,  a large
body of  mathematical forms  known as fractals,  used to  depict plant
structures, have been described with L systems \citep{kurth:99}.

The L systems  provide the sort of good  scientific abstraction needed
in plant modelling \citep[c.f.][]{regev:02}.  An L system captures the
relevant  properties   of  the  phenomena   in  its  set   of  symbols
highlighting only  the essential characteristics of the  model.  It is
computable to  support qualitative  and quantitative reasoning  of the
model  properties.    It  is  extensible,  new   symbols  can  capture
additional features of the model if required and it is understandable,
the formal  notation allows the  sharing and comparison  of scientific
knowledge.   For example,  part of  an L  system model  in  itself can
appear  in  a  publication  as  a  model  description,  unlike  models
implemented with general programming languages.

The theoretical advancements that have since been implemented in tools
based on L system formalism have been  motivated by the desire to find
out what phenomena  in plant modelling can  be  formally described and
simulated.  The range  of circumstances  where  L system  formalism is
applicable is   quite extensive  \citep{pp:99}.  \citet{kurth:99}  has
demonstrated  this  by realizing  a number  of published architectural
tree models including LIGNUM Scots pine \citep{perttunen:96} using the
L system based  tool GROGRA \citep{kurth:94}.  The simulation of plant
communities  has been reported by \citet{deussen:98}, \citet{kurth:99}
and with multiset L systems  developed by \citet{lane:02}.  Inevitably
one  has  to ask why  not  reimplement LIGNUM with   L or L+C language
\citep{karwowski:02} and make  L   systems  the basis  of   its future
development?

Plants are not closed systems since interaction with their environment
has  an important function  in their  development.  Modelling  of such
phenomena includes,  for example, computation  of the light  regime in
plant communities  and competition for growing space  (cf.  example on
bearberry,  Sec.   \ref{sec:bearberry}).   To  model  such  phenomena,
\citet{mech:97} and  \citet{mech:96} have extended  L system formalism
with communication symbols that  can pass parameter values between the
plant model in the L system and a separate program (in general purpose
language) simulating, for example, the relevant characteristics of its
environment.   \citet{kurth:94} has  implemented a  set  of predefined
functions to return  environmental information to the L  system in the
GROGRA  program.    Once  implemented,  these   separate  programs  or
predefined functions are easily used and reused.

The functionalities  enhancing L system languages make  it possible to
realize  parts of  the simulations  using general  purpose programming
languages (e.g.  unit to unit interactions) that would be difficult to
implement with the parallel rewriting  semantics.  The design of L and
L+C allows embedding of  C++, thus making such constructs unnecessary.
But note that the time spent in these environmental models is the time
spent  outside  the  L  system  formalism with  some  general  purpose
programming language.  Hence, complicated architectural tree and plant
models, such as FSTM's,  implemented using L systems inevitably employ
both an L system and  general purpose language parts.  LIGNUM combined
with  L language  also  contains those  parts,  although in  different
proportions than  models realized with L system  tools.  Therefore, as
the quest goes  on to find optimal formalisms  and modelling paradigms
to implement  complicated plant and tree models,  our current solution
is  to  mix  L  systems,  general purpose  programming  languages  and
programming libraries implementing submodels such as radiation climate
and soil properties..
 
A  further challenge  to  tree  and plant  modelling  is to  implement
source-sink  relationships  in architectural  tree  and plant  models.
Local production  and consumption of resources, which  are affected by
the environment and status of particular structural units, control the
growth of the three-dimensional structure.  Source sink phenomena have
been   modelled    by   considering   unit    to   unit   interactions
\citep{balandier:00}, accumulating information along the pathways from
root  tip   to  shoot  tip  \citep{dereffye:97}   or  solving  partial
differential  equations  \citep{deleuze:97,  palovaara:03}.   How  the
intensive calculations  required by the sink-source  approach are best
implemented in the three-dimensional plant structure is still unknown.
A  hybrid  approach  utilizing  both  L systems  and  general  purpose
languages  or other  means (e.g.   solvers of  differential equations)
offers one alternative.

\citet{coates:03} suggest that linking  empirical studies to models is
the  best way  to  provide  insight and  better  understanding of  the
implications  of the  silvicultural strategies  and the  importance of
structure  in  forest  stands.   For example  \citet{chantal:03}  have
studied the early development, size  and morphology, of Scots pine and
Norway spruce in an  experimental gap-edge environment with asymmetric
distribution of  radiation. Such experimental work  to understand tree
regeneration might benefit if the LIGNUM model was applied to describe
the study  plots, the  gap-edge zones, size  of the gaps,  the spatial
distribution  of seedlings, and  then based  on the  radiation climate
simulate the sapling development.

Finally, it would  definitely be an interesting exercise  in itself to
study carefully the  convergence between LIGNUM and L  systems and the
practical issues  involved by fully  implementing the model in  some L
system  tool, say L+C  \citep{karwowski:02}.  Some  modelling efforts,
especially  phenomena  involving single  or  directly connected  plant
parts only,  are easily expressed  by rewriting.  But  other phenomena
inevitably require modelling  and implementation outside the rewriting
formalism  simply  because  it  is  not  meaningful  to  do  so  (e.g.
computation of  the light regime)  or theoretical advancements  may be
required in  L systems, e.g.   solving for the transport  of resources
within the plant.

\section{Acknowledgments}

We   thank  Radoslaw  Karwowski   and  Przemyslaw   Prusinkiewicz  and
gratefully acknowledge their contribution  to this work.  The original
specification  of   L,  now  named  L+C,  was   jointly  developed  by
\citet{pp:99a} and,  based on this work, Radoslaw  Karwowski wrote the
original parser of L and  has further implemented the L+C language. JP
was supported by research grant No. 72569 from the Academy of Finland.
We thank two anonymous referees for their valuable comments.