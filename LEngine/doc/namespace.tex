\section{Namespace}\label{sec:namespace}  

LIGNUM  is  designed as  a  single  tree  model for  broad-leaved  and
coniferous  species.   To allow  modelling  the  development of  mixed
forest  stands  (in  the  future),  we  have  augmented  the  original
definition  of   L  with  the   concept  of  namespace   denoted  with
\texttt{open} and \texttt{close} statements (see Appendix \ref{sec:L1}
and   \ref{sec:L2}).     A   namespace   is   a    scope   or   module
\citep{stroustrup:97},  a  mechanism  to  express things  that  belong
logically together without adding too much notational burden.
 
Technically, as  each L file is  translated to C++, it  is possible to
embed C++ statements in the productions. The result of the translation
is a C++  class providing an application programming  interface to the
defined L  system.  This  includes derivation of  the string  to model
structural   development  and   the   conversion  algorithm   (Section
\ref{sec:pine}).  By implementing  namespace, each L system definition
will be enclosed  in its own unique namespace. Thus  it is possible to
design, implement and assign a specific L system to an individual tree
model realized in LIGNUM, thus  enabling the modelling of several tree
species and multiple individual trees in a single application.

