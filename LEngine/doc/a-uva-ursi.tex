\section{Bearberry}\label{sec:bearberry}

The  second example  depict the  capabilities of  L language  - LIGNUM
combination  to  deal with  various  modelling  tasks.  The  evergreen
horizontally  spreading shrub  bearberry (\it  Arctostaphylos uva-ursi
\rm L.)  grows in variety  of habitats in Northern Europe and Northern
North America.  It is a  pioneering species growing best in high light
conditions  colonizing gravelly  soils, sandy  dunes,  heathlands etc.
but   appears  also   in   the  understory   of   open  pine   forests
\citep{salemaa:99}.  It  is adapted for growth with  limited water and
nutrient supply.   Bearberry is a clonal species,  its reproduction is
mainly vegetative.

\citet{salemaa:02} have studied growth habits, axillary bud activation
and  branching  architecture of  bearberry  growing  in South  Western
Finland in  varying pollution, nutrient  and light levels.   They have
designed an L-system model  to study qualitative features of branching
patterns  of  bearberry by  simulation.   Here  their bearberry  model
growing in  sandpit is realized so  that the algorithm to  detect if a
clonal  branch  blockades  the  growth  space  of  an  active  bud  is
accomplished by LIGNUM. This also serves as an example of a stochastic
L-system  \citep{pp:90}  and   how  to  implement  global  sensitivity
\citep{kurth:94} in LIGNUM.

In  their L-system  model for  bearberry \citep{salemaa:02}  the plant
grows horizontally and consists of annual growth units holding lateral
buds  and one apical  bud.  The  buds are  divided into  dominant (D),
subdominant  (SD) and nondominant  (ND) types  \citep{remphrey:83}.  A
living bud produces a shoot of  its own type.  Axillary buds have time
delay before they release and produce a shoot.

For the implementation in L define module $B(T,S,C)$ where $T$ denotes
the bud type,  $S$ its status, i.e.  time left  before release and $C$
collision.  The  module definition for  $B$ determines the  lengths of
the shoots  produced, branching angles for lateral  shoots forking off
alternately to  the left or  to the right  and the number  of axillary
buds produced.  In  general D shoots are longer than  SD and ND shoots
and  in  the sandpit  the  colonizing  plant  shows intensive  lateral
branching due to favorable  light conditions.  Outline of the L-system
is in Appendix \ref{sec:L2}.  For details see \citet{salemaa:02}.

The  collision detection  algorithm  of LIGNUM  simply examines  given
opening angle symmetric to both sides of the growth direction of a bud
and checks if its free of other shoots and buds within given distance.
More  precisly, define $\vec  {P_1}$ as  the position  of the  bud and
$\vec D$ as its growth  direction. Define $\vec {P_2}$ the position of
the potential obstacle. Then $\vec {P_3} = \vec {P_2} - \vec {P_1}$ is
the direction from  the bud to the obstacle.   Given the opening angle
$\alpha$ and the  distance $l$ the obstacle hinders  the growth of the
bud  if $\cos(\alpha/2) =  \frac{{\vec D}  \cdot {\vec  {P_3}}} {|\vec
D||\vec {P_3}|}$ and $|\vec {P_3}| < l$.

Investigation of  whether a bud collides another  plant compartment is
implemented as pairwise comparison of structural units in LIGNUM, i.e.
not with rewriting rules in L language.  The value of parameter $C$ in
the module  $B$ is updated (Section \ref{LignumToL})  using the result
of  collision detection  calculation in  LIGNUM.  The  results  of the
development  of the  bearberry model  after 15  iterations  with three
different collision models are  in Fig.  \ref{fig:a-uva-ursi}.  As one
can  expect the  bearberry  models show  less  lateral branching  with
increasing opening angle for collision detection.
 

















