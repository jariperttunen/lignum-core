\chapter{Using Lignum}
\section{Command Line Arguments}
\section{Visualizing Results}
\subsection{The Visualization Program}
\subsection{MineSet Program}

\section{Initialization of the Tree}

The parameters and functions of the tree for the simulation 
must be given in files. The file that is given as command 
line argument is the schema file i.e., the file that gives the 
names of files where the parameters and functions are defined, 
not the parameters and functions themselves. An example of the schema file
is below:

\begin{verbatim}
#File describing the location of definition files
#for parameters and fuctions of LIGNUM
Parameters:
    Tree: Tree.txt 
    Firmament: Firmament.txt
Functions:
    FoliageMortality:FoliageMortality.fn
    Buds: Buds.fn
    DoI: DoI.fn
\end{verbatim}

The format of the schema file is simple. For the first, the \tt \# \rm
character starts a comment that extends to the end of line.  

The schema file has two main sections, one for parameters and
the second one for functions denoted by keywords 
\tt Parameters \rm and \tt Functions \rm followed by a colon.
 
The sections for parameters and functions consists pairs
composed by a keyword,(e.g., \tt Tree \rm and 
\tt Foliage Mortality\rm) and a file name
(\tt Tree.txt \rm and \tt FoliageMortality.fn \rm respectively). 
The keyword tells the purpose of the file, so that
during the intialization parameters and functions are
intialized properly. The colon is used to separate the
keyword from the file name.

If necessary, the the sections for functions and parameters
will be subdivided in the future. 

An example of the parameter file for the tree is below.

\begin{verbatim}
#Parameters for tree compartments according to papers in
#Annals of Botany 1996 and in Ecological Modelling 1998.

af      1.30        #Needle mass-tree segment area (kg/m^2)
                    #relationship
ar      0.50        #Foliage - root relationship 
lambda  1.3         #Intial value for lambda
lr      100.0       #Length - radius relationship of a 
                    #tree segment
mf      0.20        #Maintenance respiration rate of foliage
mr      0.240       #Maintenance respiration rate of roots
ms      0.0240      #Maintenance respiration rate of sapwood
na      0.7854      #Needle angle (pi/4)
nl      0.10        #Needle length (10 cm = 0.10 m) 
q       0.10        #Tree segment shortening factor
sr      0.330       #Senescence rate of roots
ss      0.07        #Senescence rate of sapwood
rho     400.0       #Density of wood in tree segment
pr      0.0010      #Proportion of bound solar radiation
                    #that is used in photosynthesis
xi      0.60        #Fraction of heartwood in tree segments
\end{verbatim}

The file simply contains parameter value pairs. Each parameter
name is reserverd keyword so that the program can recognize it.
The \tt \# \rm character begins a comment extending to the end of line.

The functions defining different behavior in the tree are given
as parametric curves in ASCII files. See the class 
\tt ParametricCurve \rm in \tt libc++adt.a \rm for details.

The keywords \tt FoliageMortality\rm, \tt Buds\rm, and
\tt DoI \rm denote functions for foliage mortality, number of new buds 
and relative shadiness (degree of interaction) respectively. More functions
will be implemented if necessary.

Currently directory paths are not parsed so all the files,
the schema file and the files defining the parameters and functions,
must be in the same directory where \lignum\ is started.
