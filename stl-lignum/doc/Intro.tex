\section{Introduction}

\lignum\ is a tree model developed by the researches 
in the Finnish Forest Research Institute, The University of Helsinki
and The Helsinki University of Technology.
The model \lignum\ has been designed to capture the three dimensional
structure of a tree species and, at the same time, to be able 
to model its growth and development.

The three dimensional structure of a tree is captured with the
help of simple structural units close to their real
world counterparts. The metabolic functioning (photo synthesis, 
respiration, transport of nutrients) is modelled by associating 
appropriate functions to these structural units. The
growth and senescence is modelled by changing the size 
of these units, creating new units and deleting these units
during a simulation.

The software that implements \lignum\ has been programmed in C++. 
Since \lignum\ was first introduced \cite{salminen:omt94, perttunen:aob96} a lot
of things have happened. During the first implementations of the model 
the terminology was not established and this inevitably
was reflected  in the software. Also, the main tool the C++ 
programming language was still under development. 
Thirdly, as characterizing any
research project many things were found by trial and error, by 
experimenting different alternatives. This too, was reflected in the software.

Today many things are different. The terminology used in \lignum\
is now established (and approved by the scientific community), 
The C++ ANSI Committee has finished its work. 
The major addition to the language from the
point of view of \lignum\ is the Standard Template Library (STL)
now providing the ubiquitous abstract datatypes needed in
software development. The research
work is still going on, but we have now established a solid set of
parameters, state variables and functions that can be
used to describe a functioning of a tree with \lignum. In addition,
\lignum\ is no longer a research project \it per se\rm, but also
used by, e.g., the students of Helsinki University of Technology.

All this leads to only one conclusion: The time has come for
a major rewrite of the software to keep The Tower of Babel from rising.       

This document describes the C++ implementation of the tree model \linebreak
\lignum. The classes, algorithms and other utilities are available in the
library called libLGM.a and related C++ header files. The
implementation uses the STL designed 
as part of ANSI C++ specification.  Before  you use the LGM library
be sure you understand STL library.

The software implementing \lignum\ is a result of an ongoing
research work, thus only a momentary (a ``snapshot'') description
can be given. The interested \lignum\ user is encouraged to report
errors (both in the software and in the documentation) to the authors. 


\section{Programming Style}
Please read the section Programming Style in the document c++adt.

