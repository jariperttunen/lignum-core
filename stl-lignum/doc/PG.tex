\chapter{Programming Guidelines}

\section{Background}
The programming language used to  implement Lignum is C++. Although it
has bias and history in systems  programming it is used by hundreds of
thousands of programmers practically in every application domain. Some
simple reasons can explain the popularity of C++. For the first it has
language  constructs  that make  it  convenient  (easy, efficient  and
reliable)  to use  many types  of modern  programming  paradigms, most
notably  object oriented  and generic  programming.  Secondly,  it has
language  facilities that  support the  implementation of  large (more
than 1000  000 line) programs.  It  is possible to  create alone small
programs containg  1000 or so lines  and make them  work through brute
force breaking each  programming paradigm and rule of  good style. But
for any  larger program  requiring cooperation of  several programmers
this  is simply not  possible.  You  will notice  that new  errors are
introduced more rapidly than the old ones are fixed.  The program will
become unmanageable.

Very few programs created today exist in one application area only. It
is very common  to have numeric applications with  at least scientific
visualization  and user  interaction  but also  data  base access  and
perhaps  even local  networking.   C++ was  not  designed for  numeric
applications but many numerical  and scientific program libraries have
been  created  in  C++  simply  because the  necessary  libraries  for
visualization and user interface already were implemented in C++. 

Lignum  is   no  exception.   The  basic   data  structures  (classes)
implementing  a generic  tree  model  don't easily  fit  into a  e.g.,
Fortran mold. Further  to be of any practical  use at least scientific
visualization must be providied with elementary user interface so that
the  results of  the  simulations  can be  studied.   Lignum has  also
evolved from a small research experiment to a medium size project with
many international  connections.  In  the future also  the programming
work to develop the model for diverse purposes will be done by several
independent  teams  of  programmers.   Hopefully  the  result  of  the
programming work done  in one team can be shared  by other teams. Thus
the maintenance of the program will  be important. C++ has been a good
choice for Lignum  and due to the development of  the language and the
introduction of  the Standard  Template Library (STL)  it is  now also
more convinient to use than ever.

\section{Using C++}

This is not  intended to be a C++ programming  manual.  There are many
excellent textbooks available.  For  example you may find \cite{Str97}
and  \cite{Brey98} useful. Stroustrup's  book not  only is  a complete
tutorial  and  reference to  the  language but  also  gives  a lot  of
practical  programming  advice   how  to  use  and  not   to  use  the
language. Breymann's book teaches the  use of STL, the essiential tool
for any serious C++ programmer.

Many of the  developers of Lignum have and  will have background other
than computer science but they have some programming experience with C
or Fortran.  Here  we have tried to collect  some most common mistakes
and pitfalls made  by the novice C++ progarmmers and  also try to give
some elementary practical  advice what to learn first  to get started.
You don't have to  learn whole C++ to work with it.  You can grow with
the language.

\subsection{C programmers}

We  have  noticed that  people  with a  strong  C  background tend  to
continue to write C with C++. It is possible to do so because C with a
few exceptions  is a subset of  C++ but the potential  benefits of the
language are lost.  Here are some common mistakes and suggestions what
to do instead.

\begin{itemize}
\item Avoid macros. Use inline functions if you want to avoid function
call overhead and for simple macros defining a constant consider const
or enum constructs.
\item Forget malloc and free. The new and delete operators do the
same job better.  Remeber to use  delete to avoid memory leaks. If you
seem to  need realloc consider  if some predefined data  structure for
example vector fulfills your  needs.  In general avoid unnecessary use
of  new  operator.  Note  that  it  is  possible to  create  temporary
variables, e.g., local to a function  on the fly whenever you need one
without using  dynamic memory allocation.  In fact only  objects (like
new tree parts)  that may have a life time of  the entire program need
to be  created dynamically  with the new  operator and stored  in some
container.
\item Avoid pointers. Use references that are safer instead as
formal parameters  and return  values. When you  have a  reference you
know  there is  a real  object you  can deal  with.  With  pointers to
create safe code you  always have to check if it is  a NULL pointer or
not. If you  have to deal with the pointers  yourself try to implement
clean interface that provides references to the user.
\item Forget C cast. If you need type cast learn to use dynamic,
static and const casts in C++. They provide type checking.
\item Use const. Everything that is constant (does not change) should
be declared const. The more you program the more you'll learn to
respect this little keyword. Do it from the beginning. It is
practically impossible to correct afterwards.
\item Avoid the use of built-in arrays and C style strings. There are
predefined string and vector classes in the STL library. They simplify
the programming and you don't loose in efficiency either. For
numerical computations there is valarray class. 
\item If you need to use unions, to do complicated pointer arithmetic, bit
shifting   or   some  other   exotic   language   fieature  hide   the
implementation deep in some function or class.  Provide clear and easy
to use  interface. But when developing  Lignum you are  unlikely to do
any of these. Lignum is a tree model and you are more likely to create
data structures  by refining  existing ones to  model tree  organs and
implement numerical methods to simulate their metabolic processes.

\end{itemize}

\subsection{Learning C++} 

The reason to learn C++ is not to learn a new syntax to do things that
has been previously done with Fortran  or C. The reason is to learn to
to design and  implement systems better and more  efficiently in a way
C++ supports the work.  This  is like learning a new natural language;
it will  take years of  practice to read  or write for  example french
fluently and correctly as a foreign language.

It is not necessary to learn C first to learn C++. In fact it might be
beneficial to learn  C++ directly without any knowledge  of C. As said
it is  not so much of  learning a new programming  language syntax but
learning  new design  and programming  paradigms.  Thus  the  top down
approach might be the best  way.  

First learn the key language  concept class and its importance in data
abstraction when  modeling the problem domain.  Learn  the concepts of
methods, friend functions, operator overloading and data members. Also
make yourself clear what is  meant by access control of class members.
People coming from natural sciences may find easy to grasp the idea of
inheritance that  is simply hierarchical ordering of  things to manage
complexity.

Then expose yourself to STL that teaches you generic programming.  The
key concept in  this paradigm is an abstract  datatype and the generic
algorithms that can  be applied to that datatype.   You will learn the
important language  concepts of templates  and functors that  are used
extensively  to implement  STL. Also  Lignum follows  the  paradigm of
generic programming.  The tree is implemented as  an abstract datatype
with a set of generic  algorithms to apply various metabolic processes
or collect information implemented as methods or functors to it.

All in  all by learning first  the high-level techniques  of C++, then
gradually the common  subset of C and C++ you'll  notice that there is
no  need or  very little  advantage to  know or  use the  most obscure
language  features.   After  all  they  are  mostly  used  by  systems
programmers  implementing operating systems,  device drivers  etc. You
will see that with C++ it  is possible to apply good programming style
to  create applications  that are  easy to  read and  understand using
program design based on sound methodology.
   
\section{Programming Style}

\subsection{Typographical Issues}
To  aid reading  and understanding  programs written  by others  a few
simple  typographical  conventions used  in  Lignum  and advice  about
writing comments are  outlined.  First, a few lines  of fictitious C++
code that provides an example of the things to consider.

\begin{verbatim}

//The purpose of the class AClass is to ...
typedef double METER;
typedef double KG;

class AClass{
friend METER AFriendFunction(AClass& obj);
public:
   AClass();
   METER firstMethodForTheClass(KG fm);
private:
   KG Wf;
   double rho;
};


//The purpose of the method is to ...
//For the physiological motivation of 
//the algorithm see ... 
METER AClass::firstMethodForTheClass(KG fm)
{
}
\end{verbatim}

The C++ class names are capitalized.   If the class name is a compound
word each word is capitalized. The  method names in a class start with
a  lower letter.   If the  method  name is  a compound  word the  each
following word is capitalized.   All (friend) functions are written as
class names, i.e.,  each word is capitalized.  User  defined types are
written in capital letters.

In Lignum we have noticed that to minimize the learning and memorizing
process  try  to be  consistent  with the  paper  published  or to  be
published when  choosing names for  variables.  For example in  a tree
segment  we  need to  have  a state  variable  for  foliage mass.   In
publications the symbol for the foliage mass is $W_f$.  So the name of
the  attribute in the  class TreeSegment  should be  \tt Wf\rm.   If a
Greek letter is used in a paper  as a symbol for a state variable or a
parameter  write it  out  as the  name  of the  class attribute.   For
example the symbol $\rho$ is used as the wood density.  So the name of
the attribute should be \tt rho\rm.   In general we use \LaTeX \ style
to write out the mathematical symbols.

Consistent  use of  indentation will  ease the  burden of  reading and
understanding of  program code.  There  are several styles in  use and
there is no reason to  prefer one over another. Selecting and learning
a text editor  like emacs the proper indentation  will become a trivia
and  in addition  to that  emacs provides  many other  useful features
improving your productivity when editing program source files.

\subsection{Comments and Documentation}
Needless to say  comments are an essential part of  a good program. But
remember misused comments can  seriously affect the readibility of the
program. Three self-evident rules of thumb for writing comments are

\begin{itemize}
  \item Comment is meaningful, i.e., the program code is not evident.
  \item Comment describes the piece of a program precisely.
  \item Comment is up to date. 
\end{itemize}
   
Don't bother to explain trivial pieces of a program. Something written
in the  language itself (like  \tt a =  b + c  \rm) does not  need any
comments. Such  comments (``a  becomes b plus  c'') just  increase the
amount of  text reader has to  go through.  Describe the  piece of the
program unambiguously. Ambiguous  and incomprehensible comments are no
use at all.  Too often when the program changes  the comments are left
untouched. Plain wrong comments are worse than no comments at all. 

Writing good comments is as difficult as writing the progam. Some
general guidelines may help to nurture this art:

\begin{itemize}
  \item Instead of explaining  the non-trivial algorithm in detail you
        can  refer  to  manuals,  text books,  publications  etc.
  \item Comment each class member and constant shortly for its
        purpose.  Comment on data is usually more useful than on algorithms.
  \item Try to answer the questions what and why rather than how. 
  \item Use ``formal comments'' to describe preconditions, postconditions 
        domain and range of a method or a function especially when
        some side effects will occur.
\end{itemize}
     
Ideally when you  return to your program after a  few weeks you should
still understand it yourself.   Tricky code should be rewritten rather
than  exhaustively commented.   Finally, comments  don't  replace the
program  documentation. In short the documentation should describe:

\begin{itemize}
  \item How to install and operate the software system.
  \item How to use the system. There is no so simple system that is
  obvious to its user.
  \item The design and the requirements of the system.
  \item The implementation and the maintenance of the system.
\end{itemize} 

You    can   study    more   about    software    documentation   from
\cite{Som85}. Ideally  the documentaiton should run  in parallell with
the programming  work.  It will mean  rewritings of the  manual due to
the changes  in the  program but any  documentation will  require many
iterations before it is useful anyway.







