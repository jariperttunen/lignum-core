\section{Algorithms}

So  far all   metabolic processes and   queries  of the  status of the
simulated  tree can be  implemented using  one  of the following  four
generic    algorithms:   ForEach,  Accumulate,   AccumulateDown    and
PropagateUp.  Each of these traverse the tree  in a slightly different
manner  and take   care that  the actual   metabolic process or  query
implemented  as a functor  is applied to  each tree compartment.  Thus
these algorithms eliminate the trouble of writing repetitive and error
prone explicit loops  each time    a  metabolic process or query    is
implemented  or refined.      The algorithms are    linear,  i.e., the
algorithmic  complexity is $O(n)$, where  $n$   is the number of  tree
compartments.

To   use these generic  algorithms  include the file Algorithms.h into
your   program.   Examples of these   algorithms   can be  obtained by
compiling  and running  the  program \tt algorithms  \rm in stl-lignum
project   directory.   To    compile    the program type    \tt   make
algorithms\rm.

The sample  tree  used in  the   examples consists  of one  main  axis
containing one tree segment, one branch whorl and the terminating bud.
The branch whorl contains two  auxilliary buds. Using the notation for
the list the tree can be written as:

\begin{displaymath}
[TS,[[B],[B]],B]
\end{displaymath}

Also the  functors used in subsequent  sections to clarify  the use of
the generic algorithms are the same as used in the sample program. The
tree segment  used in the examples  is called MyTreeSegment  and it is
simply an subclass of TreeSegment.
 
\subsection{ForEach}

ForEach is similar to  STL-library  function  for\_each.  In fact   it
applies  for\_each to  each  list  of  tree  compartments  in a  tree.
for\_each in turn simply applies a user defined functor f to each tree
compartment in a list.

\begin{description}
    \item [Signature] void ForEach(Tree<TS>\& tree, const Function\& f).
    \item [Arguments] for the function.
      \begin{description}
        \item [tree] The tree.
        \item [f]  The unary   functor  taking a   pointer to a   tree
      compartment as an argument. The functor  must return the pointer
      to a tree compartment.
     \end{description} 
   \item[Returns] The tree and tree compartments possibly modified by f.
\end{description} 

The implementation of ForEach uses   the notion function  composition.
That is, it defines functions h and g such that given the user defined
function f:

\begin{displaymath}
h = (g \circ f)(x) = g(f(x))
\end{displaymath}

The purpose of  the function g is simply to traverse  the tree.  As an
example consider  the following functor DisplayType2  playing the role
of  the function  f. The  easiest way  to ensure  proper  argument and
return    type    for    the    functor    is    to    inherit    from
AdaptableTCFunction<TS>.  It  is an  empty class but  predefines right
argument  and   return  types.  It  also  defines   (using  STL  class
unary\_function) necessary internal typedef-declarations.

\begin{verbatim}
template <class TS>
class DisplayType2: public AdaptableTCFunction<TS>{
public:
   TreeCompartment<TS>* operator ()(TreeCompartment<TS>* ts)const;
};
\end{verbatim}

The  implementation  of  the  overloaded function  operator  simply
prints out the type of the argument tree compartment.

\begin{verbatim}
template <class TS> TreeCompartment<TS>* 
DisplayType2<TS>::operator()(TreeCompartment<TS>* tc)const
{
  if (Axis<TS>* myaxis =  dynamic_cast<Axis<TS>*>(tc)){
    cout << "Hello, I'm Axis" << endl;
  }
  //checking other tree compartments similarly
  else if ...  
  ....
  return tc;
}
\end{verbatim}

The call to ForEach to print out the type of each tree compartment is
simply 

\begin{verbatim}
ForEach(tree,DisplayType2<MyTreeSegment>());
\end{verbatim}

ForEach  algorithm can be used to  implement metabolic processes e.g.,
photosynthesis   and respiration     or some  other   function    like
visualization where the computations  on   each tree compartment   are
inherently parallel.  That   is,  tree  compartments can  be   treated
individually.

\subsection{Accumulate}

The algorithm Accumulate can be used to collect data, pass information
from one tree compartment to another or query the status of a tree. It
is similar to STL-library  function accumulate.  Note however, that in
Accumulate  the initial value  (also called  the identity  element) is
passed and returned as reference,  not by value as in accumulate.  The
reason for  this design  decision is that  it is not  always numerical
data a modeller may want collect.

Accumulate traverses  the tree  and  applies the user  defined  binary
operator op to each  tree compartment with the  initial value that can
be modified according to the binary operator.

\begin{description}
   \item [Signature] T\& Accumulate(Tree<TS>\& tree, T\& init, const BinOp\& op).
   \item [Arguments] for the function.
     \begin{description}
        \item [tree] The tree.
        \item [init] The initial value. Also called the identity
     element.
        \item [op] The binary operator. The functor must take two
     arguments: the initial value and the pointer to a tree compartment. The
     operator must return the modified initial value.
     \end{description} 
   \item[Returns] The  modified initial value.
\end{description} 

More formally, the implementation of Accumulate uses the notion of
function composition. It defines functions h and g such that given the
user defined function f:

\begin{displaymath}
h = (g \circ f)(x,y) =  g(f(x,y),y)
\end{displaymath}

In practice the x is the  identity and y  is the tree compartment. The
purpose of the function g  is simply to  traverse the tree and it does
not modify the identity element or the tree compartment. As an example
of   the use  of  Accumulate  consider  the  following binary operator
CountCompartments:

\begin{verbatim}
template <class TS>
class CountCompartments{
public:
  int& operator ()(int& id,TreeCompartment<TS>* ts)const;
};
\end{verbatim}

The  implementation   of   the  overloaded   function     operator  in
CountCompartments simply  counts the number of  tree compartments in a
tree and echos the type of each compartment.

\begin{verbatim}
template <class TS>
int& CountCompartments<TS>::operator()(int& n,TreeCompartment<TS>* tc)const
{
  if (Axis<TS>* myaxis =  dynamic_cast<Axis<TS>*>(tc)){
    cout << "Hello, I'm Axis ";
  }
  //checking other tree compartments similarly
  else if ...  
  ....
  n+=1;
  return n;
}
\end{verbatim}

To count all tree compartments in a tree one simply calls Accumulate
with the functor CountCompartments.

\begin{verbatim}
int i = 0;
int n = Accumulate(tree,i,CountCompartments<MyTreeSegment>());
\end{verbatim}

The  algorithm   Accumulate  can  be   used,  e.g.,  to  sum   up  the
photosynthates and respiration of all tree segments in a tree. Another
example  could be a  functor that  collects all  tree segments  into a
sequence.  The computations \it must not assume \rm in which order the
tree  compartments  are   traversed.   Algorithms  AccumulateDown  and
PropagateUp can be  used to traverse the tree  explicitely from top to
bottom and from below upwards respectively.

\subsection{AccumulateDown}

The AccumulateDown  algorithm can also  be used to collect  data, pass
information from one tree compartment to anoter or query the status of
a tree.  However, as the name  suggests the tree is traversed from the
tip of the branches  to the base of the tree.  More  over each axis is
traversed  independently (semantically in  parallell).  Note  that the
initial  value  is  passed  and  returned  as  reference  because  the
computations are not necessarily numerical.

AccumulateDown traverses the tree and  applies the user defined binary
operator op  to each tree compartment  with the initial value that can
be modified according to the binary operator.

\begin{description}
   \item [Signature] T\& AccumulateDown(Tree<TS>\& tree,T\& init, const BinOp\& op).
   \item [Arguments] for the function.
     \begin{description}
        \item [tree] The tree.
        \item [init] The initial value. Also called the identity
     element.
        \item [op] The binary operator. The functor must take two
     arguments: the initial value and the pointer to a tree compartment. The
     operator must return the modified initial value.
     \end{description} 
   \item[Returns] The  modified initial value.
\end{description} 

Because  each axis is   traversed   independently the results of   the
compuations  are  summed up in  branching   points.  Thus each initial
element must have the overloaded add and  assign (+=) operator defined
with  appropriate semantics. Also a  new initial  value is created for
each  axis  so  the call to   a  constructor  must  be  possible. More
precisely, if  the type of   the initial value  is  T the call to  the
constructor T() must succeed.

To  count  the   number of   tree  segments  with  AccumulateDown  and
CountCompartments one simply writes:

\begin{verbatim}
int i = 0;
int n = AccumulateDown(tree,i,CountCompartments<MyTreeSegment>());
\end{verbatim}

Assuming  that the  add and  assign operator  and the  constructor are
defined for  the identity element\footnote{Note that  they are defined
for  the integer  and floating  types.} one  can use  AccumulateDown as
Accumulate.   However, some  metabolic processes  are depended  on the
information  flowing downwards  in  a tree.   These  processes can  be
implemented only with the  AccumulateDown algorithm.  One such process
is diameter growth.  Other examples  can be various models for hormone
and nutrient flows downwards in a tree.

\subsection{PropagateUp}

The algorithm PropagateUp  can be thought as  an inverse  operation to
AccumulateDown. The tree is traversed from the  base to the tip of the
axis. During the traversal the initial value is modified by the binary
operarator. Further  more each axis  is computed  independently, i.e.,
inherently in parallell. In each branching point a copy of the current
value or status of the  initial element is  sent forward to each axis.
Note again  that the initial value  is passed as  reference. It is not
always numerical data one wants to send up in a tree.

\begin{description}
   \item [Signature] void PropagateUp(Tree<TS>\& tree,T\& init, const BinOp\& op).
   \item [Arguments] for the function.
     \begin{description}
        \item [tree] The tree.
        \item [init] The initial value. Also called the identity
     element.
        \item [op] The binary operator. The functor must take two
     arguments: the initial value and the pointer to a tree compartment. The
     operator must return the pointer to a tree compartment.
     \end{description} 
   \item[Returns] Nothing.
\end{description} 
 
Because  each  axis is computed   independently a copy of  the current
value  of  the initial  element   is  passed forward   in a  branching
point. Thus the overloaded assignment operator (=) must be defined for
the initial element.

As an example of simple signal passing in a  tree consider the functor
MyExampleSignal. It receives an signal as in integer value and at each
tree compartment it increases signal's value by one.

\begin{verbatim}
template <class TS> TreeCompartment<TS>* 
MyExampleSignal<TS>::operator()(int& n,TreeCompartment<TS>* tc)const
{
  if (Axis<TS>* myaxis =  dynamic_cast<Axis<TS>*>(tc)){
    cout << "Hello, I'm Axis ";
  }
  //checking other tree compartments similarly
  else if ...  
  ....
  n+=1;
  return tc; 
}
\end{verbatim}

To pass a signal upwards in a tree starting from zero one simply calls
PropagateUp with MyExampleSignal.

\begin{verbatim}
int s = 0;
PropagateUp(tree,s,MyExampleSignal<MyTreeSegment>());
\end{verbatim}

Other possible  applications for  PropagateUp   in addition to  signal
passing can be, e.g., modelling the water flow upwards in a tree.


