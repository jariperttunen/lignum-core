
\section{PositionVector}
The class \tt PositionVector \rm provides a simple 
position vector in a three dimensional right handed coordinate system 
with $z$-axis pointing upwards, $x$-axis pointing to the south and 
$y$-axis pointing to the east (Figure ~\ref{fig:pv}). 
This class can also be used as a direction 
vector when the length of the vector is 1. 

The user can (after given initial coordinates) rotate the 
vector about all the three axes and query the directions cosines $\alpha$,
$\beta$ and $\gamma$.

The length of the vector must be greater than $0$. Otherwise, by
definition of direction cosines, floating point errors will occur during
the operations.

\begin{figure}[ht]
\begin{picture}(300,250)
%\graphpaper(0,0)(300,250)
\put(150,100){\vector(2,3){55}}

\put(150,100){\vector(0,1){100}}
\put(150,100){\vector(-2,-1){80}}
\put(150,100){\vector(1,0){100}}
\put(70,45){$X$}
\put(255,100){$Y$}
\put(150,205){$Z$}

\put(200,190){$(x,y,z)$}

\qbezier(140,95)(145,120)(158,113)
\qbezier(160,115)(170,115) (170,100)
\qbezier(150,120)(158,128) (162,119)

\put(135,110){$\alpha$}
\put(170,115){$\beta$}
\put(155,128){$\gamma$}
\end{picture}
\caption{Position Vector in a Right Handed Coordinate System}\label{fig:pv}
\end{figure}

\subsection{Using PositionVector}
The \tt PositionVector \rm abstracts the rotation matrices and their
operations and provides a simple interfaces to define rotations
in three dimensional coordinate system. Assuming that \tt pi \rm 
is a symbolic value for $\pi$ the following sequence of operations 
define a default position vector and rotates it to position $(0,1,0)$.

\begin{tabbing}
\tt
Tabbing \= Tabbing \= Tabbing \= \kill
\>\>\>\tt PositionVector pv; \\
\>\>\>\tt pv.rotate(ROTATE\_Y,pi/2.0); \\
\>\>\>\tt pv.rotate(ROTATE\_Z,pi/2.0);
\end{tabbing}

The declaration of \tt pv \rm defines a default vector at position $(0,0,1)$.
The first 90 degree rotation about $y$-axis moves the end point to position 
$(1,0,0)$. Finally, the second 90 degree rotation about $z$-axis 
moves the end point to the position $(0,1,0)$.
 
\subsection{The Class Declarataion of PositionVector}
\begin{verbatim}
 
class PositionVector{
  friend double Dot(const PositionVector& p1, const PositionVector& p2);
  friend PositionVector Cross(const PositionVector& p1, 
                              const PositionVector& p2);  
public:
  PositionVector();
  PositionVector(const double x, const double y, const double z);
  PositionVector(const PositionVector& pv);
  PositionVector(const vector<double>& v1);
  PositionVector& rotate(ROTATION direction, RADIAN angle);
  PositionVector& operator = (const PositionVector& pv);
  double length()const;
  double alpha()const;
  double beta()const;
  double gamma()const;
  vector<double> getVector()const{return v;}
  PositionVector& normalize(); 
private:
  vector <double> v;
};
\end{verbatim}

\subsection{Functions and Operations}
\subsubsection{Dot}
The  dot product of two vectors.

\subsubsection{Cross}
The cross product of two vectors.

\subsection{Public Methods}
\subsubsection{PositionVector()}
Default constructor creating vector to the point $(0,0,1)$. 
    \begin{description}
       \item [Returns] Nothing.
    \end{description} 

\subsubsection{PositionVector(const double x, const double y,\\ 
const double z);}
Constructor creating a position vector to the point $(x,y,z)$. 
    \begin{description}
       \item [Returns] Nothing.
    \end{description} 

\subsubsection{PositionVector(const vector<double>\& v);} 
Constructor create a position vector to point $(v[0],v[1],v [2])$. 
    \begin{description}
       \item [Returns] Nothing.
    \end{description} 

\subsubsection{PositionVector\& rotate(ROTATION d, RADIAN a);} 
Rotate the vector about the  axis specified by \tt d \rm by angle
specified by \tt a\rm.
 \begin{description}  
    \item [Arguments] for the method.
      \begin{description}
        \item [d] The axis of rotation. Possible values are \\
                  \tt ROTATE\_X\rm, \tt ROTATE\_Y \rm
                  or \tt ROTATE\_Z \rm for $x$, $y$ and 
                  $z$-axis respectively. 
        \item [a] The angle of rotation in radians.
      \end{description}
    \item [Returns] The rotated position vector itself.
 \end{description}

\subsubsection{PositionVector\& operator = (const PositionVector\& pv);}
The assignment operator. 
    \begin{description}
       \item [Returns] The position vector on the left side of the assignment.
    \end{description}

\subsubsection{double length()const;}
 Compute the length of the position vector.
   \begin{description}
       \item [Returns] The length of the vector.
   \end{description} 

\subsubsection{double alpha()const;} 
Compute the direction cosine $\alpha$ (Figure \ref{fig:pv}) between the position 
vector and $x$-axis.
  \begin{description}
       \item [Returns] The angle $alpha$  defined as $\arccos (x / l)$
                       where $l$ is the length of the vector.
   
  \end{description} 

\subsubsection{double beta()const;} 
Compute the direction cosine $\beta$ (Figure \ref{fig:pv}) between the position 
vector and $y$-axis.
  \begin{description}
       \item [Returns] The angle $\beta$  defined as $\arccos (y / l$)
                       where $l$ is the length of the vector. 
  \end{description} 

\subsubsection{double gamma()const;}
Compute the direction cosine $\gamma$ (Figure \ref{fig:pv}) between the position 
vector and $z$-axis.
  \begin{description}
       \item [Returns] The angle $\gamma$ defined as $\arccos (z / l)$
                       where $l$ is the length of the vector.
  \end{description} 

 
\subsubsection{vector<double> getVector()const;} 
Return a copy of the internal vector containing the end points 
of the position vector. 
  \begin{description}
       \item [Returns] The copy of the internal vector. 
  \end{description} 


\subsubsection{normalize}
Normalizes the vector. The vector can be then used as a direction
vector, i.e., the length of the vector will be 1 (one).

\subsection{Private Data Members}
\begin{description}
 \item [vector<double> v] The end point of the position vector.

 \end{description} 



