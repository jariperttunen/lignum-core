\section{CString}
\bf Note: \rm this class is obsolete. Please study and use the STL
class string instead. \linebreak

The class \tt CString \rm implements a string. The most
useful property compared to the basic built in \tt char* \rm
data type in C/C++ is that the memory management is abstracted 
from the user.

The  operations on  the  string  class  are similar  to the operations
provided by the standard  C-library. Not all the functions in the C-library
are implemented, only those that are ubiquitos when using 
strings. But the constructors  and  type cast
operations  provide transparent interface for \tt char* \rm data
type and consequently to the standard C-library string functions. 

\subsection{The Class Declaration for CString}
\begin{verbatim}
class CString{
  friend ostream& operator<<(ostream&,
                             const CString& s1);
  friend CString operator+(const CString& s1, 
                           const CString& s2);
  friend inline int operator!=(const CString& s1, 
                               const CString& s2);
  friend inline int operator==(const CString& s1, 
                               const CString& s2);
public:
  inline CString(const char * str= "");
  inline CString(const CString& str);
  ~CString();
  CString& operator=(const CString& str);
  CString& operator+=(const CString& str);
  int length()const {return str_length;}
  inline CString strStr(const CString& str)const;
  CString subStr(unsigned int start, 
                 unsigned int end)const;
  inline CString& toLower();
  inline int isSpace();
  operator const char*()const;
  operator const char*();
  operator char*();
private:
  char *string;
  int str_length;
};
\end{verbatim}

\subsection{Public Methods}

\subsubsection{inline CString(const char* str= "");}
Default constructor that creates a string from \tt char* \rm datatype. 

\begin{description}
 \item [Arguments] to the method.
   \begin{description} 
     \item [str] The C character vector. Default value is an empty string.
   \end{description}
 \item [Returns] Nothing.
\end{description}

\subsubsection{inline CString(const CString\& str);}
Copy constructor.

\subsubsection{\~{ }CString();}
Destructor, releases the memory allocated.

\subsubsection{CString\& operator=(const CString\& str);}
The assignment operator.
\begin{description}
 \item [Returns] The modified string itself (on the left side of the 
                  assignment).
\end{description}

\subsubsection{CString\& operator+=(const CString\& str);}
The concatination operator. 
\begin{description}
 \item [Returns] The modified string itself (on the left side side of the 
                  concatination).
\end{description}

\subsubsection{int length()const;} 
Query the length of the string.
\begin{description}
 \item [Returns] The length of the string.
\end{description}

\subsubsection{inline CString strStr(const CString\& str)const}
Check if \tt str \rm is a substring of the string method is
applied to as defined by C-library function \tt strstr\rm. 
 \begin{description}
    \item [Arguments] for the method.
      \begin{description}
        \item [str] A string, possibly empty.
      \end{description}
    \item [Returns] If \tt str \rm is not a substring return empty string. 
                    Othewise return a string that starts from the
                    first occurence of the substring. For example, 
                    if \tt str \rm is ``needle'' and the string the method is 
                    applied to is ``findaneedleinatree'' then
                    ``needleinatree'' is returned.
 \end{description}

\subsubsection{CString subStr(unsigned int start, unsigned int end)const;}
Create a substring from the position \tt start \rm to the position 
\tt end\rm.

 \begin{description}
    \item [Arguments] for the method.
      \begin{description}
        \item [start] Starting position, such that 0 $\leq$ \tt start \rm 
                      $\leq$ length-1.
        \item [end] End position, such that \tt start \rm  $\leq$
                    \tt end \rm $\leq$ length-1.
      \end{description} 
      \item[Returns] The new substring.
 \end{description} 
  
\subsubsection{inline CString\& toLower();}
Converts characters in the string to lowercase using C-library function 
\tt tolower\rm. 
 \begin{description}
    \item [Returns] The modified string in lowercase. 
 \end{description} 
 
\subsubsection{inline int isSpace();}
Test if the string contains only whitespace
characters using C-libary function \tt isspace\rm. 
 \begin{description}
    \item [Returns]  true or false in terms of C (i.e, false 
                     is defined as 0 and true anything else).
 \end{description}

\subsubsection{type casts}
The three type cast operations create transparent interface to the functions 
manipulating strings in C-library.

\begin{tabbing}
Tabbing \= Tabbing \= Tabbing \= \kill
\>\>\>\tt  operator const char*()const; \\ 
\>\>\>\tt  operator const char*(); \\
\>\>\>\tt  operator char*();
\end{tabbing}

 \begin{description}
    \item [Returns] The character vector (\tt char*\rm) containing the string.
  \end{description}

\subsection{Private Data Members}
 \begin{description}
    \item [char *string] The character vector containing the string. 
    \item [int str\_length] The length of the string.
\end{description}

\subsection{Operations and Functions on CString}

\subsubsection{friend ostream\& operator<<(ostream\& os, 
                                           const CString\& str);}
Print the string to standard output (\tt cout\rm).
 \begin{description}
   \item [Arguments] to the operation.
     \begin{description}
       \item [os] The standard output stream (\tt cout \rm in a normal case).
       \item [str] The string.
     \end{description}
   \item [Returns] The output stream.
 \end{description}


\subsubsection{friend CString operator+(const CString\& s1,\\
                                        const CString\& s2);}
Concatinate strings \tt s1 \rm and \tt s2\rm. Strings \tt s1 \rm and
\tt s2 \rm are not changed. 
 \begin{description}
   \item [Returns] The new string.
 \end{description}

\subsubsection{friend inline int operator!=(const CString\& s1,\\
                                            const CString\& s2);}
Test if the two string are \it not \rm equal.

\begin{description}
   \item [Returns] true or false in terms of C (i.e., 0 is false and
                   true is anything else).
\end{description}

\subsubsection{friend inline int operator==(const CString\& s1,\\
                                              const CString\& s2);}
Test if the two strings are equal. 
\begin{description}
   \item [Returns] true or false in terms of C (i.e., 0 is false and
                   true is anything else).
\end{description}
