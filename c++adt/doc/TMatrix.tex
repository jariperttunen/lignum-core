\section{TMatrix}

The \tt TMatrix  \rm implements an $n \times  m$ matrix. Currently the
basic methods and operations  for the matrices are implemented.  These
include  assignment, indexing,  negating the  elements,  transpose and
multiplications  with another  matrix  and a  vector.  Another set  of
methods   implement  memory   management.   These   are  constructors,
destructor, assignment and resizing the matrix.

For the  matrix operations (e.g., multiplication,  addition) the sizes
of the  matrices is checked  before performing the operation.   If the
operation cannot be performed error message is printed and the program
exits.

\subsection{The Class Declaration of TMatrix}
\verbinput{../include/TMatrix.h}

\subsection{Template Parameters}

\begin{description}
  \item[T] The type of the matrix element. Any of the 
   arithmetic types in C/C++ can be used. Technically speaking, any type that
   implements the set of friend functions for addition and
   multiplications can be the matrix element.
\end{description}

\subsection{Public Methods}

\subsubsection{Matrix}
The overloaded  constructor for the  matrix. The first one  creates an
undefined default  matrix.  Use  \tt resize \rm  method to  define the
size of the matrix before using it.
\begin{description}
  \item[Signature 1] Matrix().
  \item[Signature 2] Matrix(const Matrix<T>\& m);
  \item[Arguments] for the method.
     \begin{description}
        \item[m] The matrix.
     \end{description}  
  \item[Signature 2] Matrix(const int r,const int c).
  \item[Arguments] for the method.
     \begin{description}
        \item[r] Number of rows in the matrix.
        \item[c] Number of columns in the matrix.
     \end{description}
  \item[Returns] Nothing.
\end{description}


\subsubsection{\~{ }TMatrix}
Destructor releases the memory allocated for the matrix.

\subsubsection{operator=}
The assignment operator.
\begin{description}
   \item[Signature] TMatrix<T>\& operator=(const TMatrix<T>\& m).
   \item[Arguments] for the method.
      \begin{description}
          \item[m] The matrix on the right side of the assignment.
      \end{description}
   \item[Returns] The matrix itself on the left side of the assignment.
\end{description}

\subsubsection{operator[]}

The overloaded  indexing operator returning  a row in a  matrix.  Note
that  the row  is returned  as  a built  in C/C++  vector type.   This
operator is seldom  used alone but it is  usually followed by indexing
also  the vector  returned returning  one element  in the  matrix. For
example, \tt  m[0][2] \rm  gives the element  in the first  row, third
column. Thus the indexing follows the C/C++ vector indexing.

\begin{description}
   \item[Signature] T* operator[](int r)const.
   \item[Arguments] to the method.
    \begin{description}
      \item[r] The row in the matrix that must be 
                 $0 \geq$ \tt row $\geq$ \tt rows() \rm $- 1$.
      \end{description}
  \item[Returns] The row in the matrix as a built in C/C++ vector type.
                 C/C++ style of indexing is used, i.e., the first row is 0, 
                 the second 1 etc. 
\end{description}

\subsubsection{operator-}
Negate the elements of the matrix. 
\begin{description}
    \item[Signature] TMatrix<T>\& operator -().
   \item[Returns] The matrix itself with the elements negated.
\end{description}

\subsubsection{transpose}
The transpose of the matrix. 

\begin{description}
    \item[Signature] TMatrix<T> transpose()const.
   \item[Returns] New matrix as a transpose of the matrix
                  the method is applied to.
\end{description}

\subsubsection{rows}
The number of rows in the matrix.

\begin{description}
    \item[Signature] int rows()const.
    \item[Returns] The number of rows.
\end{description}

\subsubsection{cols}
The number of columns in the matrix.

\begin{description}
   \item[Signature] int  cols()const.
   \item[Returns] The number of columns.
\end{description}

\subsubsection{resize}
Release the memory for the matrix and resize it.
\begin{description}
   \item[Signature] void resize(const int r, const int c).
   \item[Arguments] for the method.
     \begin{description}
       \item[r] The number of rows in the new matrix.
       \item[c] The number of columns in the new matrix.
     \end{description}
   \item[Returns] Nothing.
\end{description}

\subsection{Functions and operations on TMatrix}
\subsubsection{operator+}
Add two $n \times m$ matrices $A$ and $B$.
\begin{description}
   \item[Signature] friend TMatrix<T> operator+(const TMatrix<T>\& m1,const TMatrix<T>\& m2) 
  \item [Arguments] for the method.
    \begin{description}
      \item [m1] The  $n \times m$ matrix $A$.
      \item [m2] The  $n \times m$ matrix $B$.
    \end{description}
  \item [Returns] New  $n \times m$ matrix $C$ such that $C = A + B$.
\end{description}

\subsubsection{operator*}
Multiply $n \times m$ matrix $A$ with $m \times k$ matrix $B$.
\begin{description} 
  \item[Signature] friend TMatrix<T> operator*(const TMatrix<T>\& m1,
  const TMatrix<T>\& m2).
  \item[Arguments] for the method.
    \begin{description}
      \item [m1] The  $n \times m$ matrix $A$.
      \item [m2] The  $m \times k$ matrix $B$.
    \end{description}
   \item [Returns] New $n \times k$ matrix $C$ such that $C = A \times B$.
\end{description}

\subsubsection{operator*}

Multiply a row vector $v$ of length $n$ with $n \times m$ matrix $A$.
The \tt vector<T> \rm is a vector type in STL.
\begin{description}
  \item[Signature] friend vector<T> operator*(const vector<T>\&
   v,const TMatrix<T>\& m).
  \item[Arguments] for the method.
   \begin{description}
     \item [v] The vector $v$ of length $n$.
     \item [m] The $n \times m$ matrix $A$.
   \end{description} 
  \item [Returns] New vector $k$ of length $m$ such that  $k = v \times A$.
\end{description} 


\subsubsection{operator*}
Multiple an $n \times m$ matrix $A$ with a column vector $v$ of length $m$. 
Result is vector $k$ of size $n$. The \tt vector<T> \rm is a vector 
type in STL.

\begin{description}
  \item[Signature] friend vector<T> operator*(const TMatrix<T>\& m,const vector<T>\& v).
  \item[Arguments] for the method.
   \begin{description}
     \item [m] The $n \times m$ matrix $A$.
     \item [v] The vector $v$ of length $n$.
   \end{description} 
  \item [Returns] New vector $k$ of length $n$ such that  $k = A \times v$.
\end{description}

\subsubsection{friend TMatrix<T> operator*(const TMatrix<T>\& m,\\
    	                                   const T scalar);}
Multiply each element in a matrix by a scalar.
\begin{description}
  \item[Signature 1] friend TMatrix<T> operator*(const TMatrix<T>\&
  m,const T s).
  \item[Signature 2] friend TMatrix<T> operator * (const T s,const
  TMatrix<T>\& m).
  \item[Arguments] for the method.
   \begin{description}
     \item [m] The $n \times m$ matrix.
     \item [s] The scalar.
   \end{description} 
  \item [Returns] New matrix.
\end{description}

\subsubsection{operator<<}
Print the matrix to an output stream (usually the standard output \tt cout\rm).

\begin{description}
  \item[Signature] friend ostream\& operator << (ostream\& os,const TMatrix<T>\& m).
  \item[Arguments] for the method.
   \begin{description}
     \item [os] The output stream (usually \tt cout\rm).
     \item [m] The $n \times m$ matrix.
   \end{description} 
  \item [Returns] The output stream.
\end{description}

\subsection{The Private Data Members}

\begin{description}
  \item [matrix\_table] The $n \times m$ vector for the  
                           $n \times m$ matrix.
  \item [n\_of\_rows] The number of rows in the matrix.
  \item [n\_of\_cols] The number of columns.
\end{description}

