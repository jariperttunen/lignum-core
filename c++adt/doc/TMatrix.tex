\section{TMatrix}

The \tt TMatrix \rm implements an $n \times m$ matrix. Currently
the only methods and operations implementing matrix operations 
are assignment, indexing, negating the elements, transpose and 
multiplications with another matrix and a vector.

Another set of methods impelement memory management. These
are constructors, destructor, assignment  and resizing the matrix.

For the matrix operations (e.g., multiplication, addition)
 the sizes of the matrices is checked
before performing the operation. 
If the operation cannot be performed
error message is printed and the program exists.

\subsection{The Class Declaration of TMatrix}
\begin{verbatim}
template <class T> class TMatrix {
  friend TMatrix<T> operator+(const TMatrix<T>& m1,
                              const TMatrix<T>& m2);
  friend TMatrix<T> operator*(const TMatrix<T>& m1,
                              const TMatrix<T>& m2);
  friend vector<T> operator*(const vector<T>& v, 
                             const TMatrix<T>& m);
  friend vector<T> operator*(const TMatrix<T>& m, 
                             const vector<T>& v);
  friend TMatrix<T> operator*(const TMatrix<T>& m,
                              const T scalar);
  friend TMatrix<T> operator*(const T scalar,
                              const TMatrix<T>& m);
  friend ostream& operator<<(ostream& os, 
                             const TMatrix<T>& m);
public:
  TMatrix();    
  TMatrix(const int rows,const int cols);
  TMatrix(const TMatrix<T>& m);
  ~TMatrix();
  TMatrix<T>& operator = (const TMatrix<T>& m);
  T* operator[](int row)const;
  TMatrix<T>& operator -();
  TMatrix<T> transpose()const;
  int rows() const;
  int cols() const;
  void resize( const int rows, const int cols);  
private:
  T* matrix_table;
  int n_of_rows;
  int n_of_cols;
};

\end{verbatim}

\subsection{Template Parameters}

\begin{description}
  \item[T] The type of the matrix element, any of the 
   arithmetic types in C/C++.
\end{description}

\subsection{Public Methods}

\subsubsection{Matrix()}
The default constructor creates a default matrix. You have to use 
\tt resize() \rm method to define the size of the matrix 
before using it. 
\begin{description}
  \item[Returns] Nothing.
\end{description}

\subsubsection{Matrix(const int rows,const int cols);}
Constructor creating $n \times m$ matrix.
\begin{description}
  \item[Arguments] to the method.
    \begin{description}
      \item[rows] The number of rows in the matrix.
      \item[cols] The number of columns in the matrix.
    \end{description}
\end{description}

\subsubsection{Matrix(const Matrix<T>\& m);}
The copy constructor.

\begin{description}
  \item[Returns] Nothing.
\end{description}

\subsubsection{\~{ }TMatrix()}
Destructor releases the memory allocated for the matrix.

\begin{description}
  \item[Returns] Nothing.
\end{description}

\subsubsection{TMatrix<T>\& operator=(const TMatrix<T>\& m);}
The assignment.

\begin{description}
  \item[Returns] The matrix itself on the left side of the assignment.
\end{description}

\subsubsection{T* operator[](int row)const;}

The overloaded indexing operator returning a row in a matrix.
Note that the row is returned as a built in C/C++ vector type.
This operator is seldom used alone but it is usually followed
by indexing also the vector returned returning one  element in the
matrix. For example, \tt m[0][2] \rm gives the element in the
first row, third column.

\begin{description}
  \item[Arguments] to the method.
    \begin{description}
      \item[row] The row in the matrix, such that 
                 $0 \geq$ \tt row $\geq$ \tt rows() \rm $- 1$.
      \end{description}
  \item[Returns] The row in the matrix as a built in C/C++ vector type.
                 C/C++ style of indexing is used, i.e., the first row is 0, 
                 the second 1 etc. 
\end{description}

\subsubsection{TMatrix<T>\& operator -();}
Negate the elements of the matrix. 
\begin{description}
   \item[Returns] The matrix itself with the elements negated.
\end{description}

\subsubsection{TMatrix<T> transpose()const;}
The transpose of the matrix. The 

\begin{description}
   \item[Returns] New matrix as a transpose of the matrix
                  the method is applied to.
\end{description}

\subsubsection{int rows()const;}
The number of rows in the matrix.

\begin{description}
   \item[Returns] The number of rows.
\end{description}

\subsubsection{int cols()const;}
The number of columns in the matrix.

\begin{description}
   \item[Returns] The number of columns.
\end{description}

\subsubsection{void resize(const int rows, const int cols);}
Release the memory for the matrix and resize it.
\begin{description}
   \item[Arguments] for the method.
     \begin{description}
       \item[rows] The number of rows in the new matrix.
       \item[cols] The number of columns in the new matrix.
     \end{description}
   \item[Returns] Nothing.
\end{description}

\subsection{Functions and operations on TMatrix}
\subsubsection{friend TMatrix<T> operator+(const TMatrix<T>\& m1, \\
                                           const TMatrix<T>\& m2);}
Add two $n \times m$ matrices $A$ and $B$.
\begin{description}
  \item [Arguments] for the method.
    \begin{description}
      \item [m1] The  $n \times m$ matrix $A$.
      \item [m2] The  $n \times m$ matrix $A$.
    \end{description}
  \item [Returns] New  $n \times m$ matrix $C$ such that $C = A + B$.
\end{description}

\subsubsection{friend TMatrix<T> operator*(const TMatrix<T>\& m1,\\
                                            const TMatrix<T>\& m2);}
Multiply $n \times m$ matrix $A$ with $m \times k$ matrix $B$.
\begin{description} 
  \item[Arguments] for the method.
    \begin{description}
      \item [m1] The  $n \times m$ matrix $A$.
      \item [m2] The  $m \times k$ matrix $B$.
    \end{description}
   \item [Returns] New $n \times k$ matrix $C$ such that $C = A \times B$.
\end{description}

\subsubsection{friend vector<T> operator*(const vector<T>\& v,\\
                                          const TMatrix<T>\& m);}
Multiply a row vector $v$ of length $n$ with $n \times m$ matrix $A$.
The \tt vector<T> \rm is a vector type in STL.
\begin{description}
  \item[Arguments] for the method.
   \begin{description}
     \item [v] The vector $v$ of length $n$.
     \item [m] The $n \times m$ matrix $A$.
   \end{description} 
  \item [Returns] New vector $k$ of length $m$ such that  $k = A \times v$.
\end{description} 


\subsubsection{friend vector<T> operator*(const TMatrix<T>\& m,\\
                                          const vector<T>\& v);}
Multiple an $n \times m$ matrix $A$ with a column vector $v$ of length $m$. 
Result is vector $k$ of size $n$. The \tt vector<T> \rm is a vector 
type in STL.

\begin{description}
  \item[Arguments] for the method.
   \begin{description}
     \item [m] The $n \times m$ matrix $A$.
     \item [v] The vector $v$ of length $n$.
   \end{description} 
  \item [Returns] New vector $k$ of length $n$ such that  $k = v \times A$.
\end{description}

\subsubsection{friend TMatrix<T> operator*(const TMatrix<T>\& m,\\
    	                                   const T scalar);}
Multiply each element in a matrix by a scalar.

\begin{description}
  \item[Arguments] for the method.
   \begin{description}
     \item [m] The $n \times m$ matrix.
     \item [scalar] The scalar.
   \end{description} 
  \item [Returns] New matrix.
\end{description}

\subsubsection{friend TMatrix<T> operator * (const T scalar,\\
	                                     const TMatrix<T>\& m);}
Multiply each element in a matrix by a scalar.

\begin{description}
  \item[Arguments] for the method.
   \begin{description}
     \item [scalar] The scalar.
     \item [m] The $n \times m$ matrix.
   \end{description} 
  \item [Returns] New matrix.
\end{description}

\subsubsection{friend ostream\& operator << (ostream\& os, \\
                                            const TMatrix<T>\& m);}
Print the matrix to an output stream (usually the standard output \tt cout\rm).

\begin{description}
  \item[Arguments] for the method.
   \begin{description}
     \item [os] The output stream (usually \tt cout\rm).
     \item [m] The $n \times m$ matrix.
   \end{description} 
  \item [Returns] The output stream.
\end{description}

\subsection{The Private Data Members}

\begin{description}
  \item [T* matrix\_table] The $n \times m$ vector for the  
                           $n \times m$ matrix.
  \item [int n\_of\_rows] The number of rows in the matrix.
  \item [int n\_of\_cols] The number of columns.
\end{description}

