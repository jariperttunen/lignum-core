
\section{ParametricCurve}

The class \tt ParametricCurve \rm can be used to define simple
functions.  The approximation of a function
can be described  in an ASCII file that contains a set of $(x,y)$ pairs 
defining the values of the function at given points. 
There is no limit to the number of points
that can be defined. At least two points must be defined.

The $x$ values must be in ascending
order and each $x$ value must have exactly one $y$ value. 
Functions having steps can be defined by having two
close enough $x$ values. The closer the two $x$ values are
the better the step in the function is
described.

The values of the function between the defined pairs are 
interpolated  with a straight line. Values outside the defined
pairs are extrapolated with a straight line defined by
first two or last two$(x,y)$ pairs  found in the file.

\subsection{Using Parametric Curve}
 
An example of a file and the parametric curve it defines 
is in Figure~\ref{fig:curve}. Unfortunately the current implementation
does not allow the user to include any comments in the file. 

Assuming the name of the file is \tt curve.fn \rm the following 
sequence of operations reads the file and evaluates the value of the
curve at $x = 2.0$.
\begin{tabbing}
Tabbing \= Tabbing \= Tabbing \= Tab \= \kill
\>\>\>\tt  ParametricCurve pc(``curve.fn''); \\
\>\>\>\tt  if (pc.ok()) \\
\>\>\>\>\tt pc.eval(2.0);
\end{tabbing}
First, the parametric curve is created and then after checking
if the program was able to read the file the value of the curve
is evaluated.

\begin{figure}[h]
\begin{center}
\begin{picture}(330,250)
%\graphpaper(0,0)(330,250)
\put(0,0){
  \begin{picture}(50,250)
  \put(55,220) {File ``curve.fn''}

  \put(50,160){\begin{tabular}{r r} 
                    -7.0 & -3.0 \\ 
                    -1.0 &  0.0 \\ 
                     2.0 &  5.0 \\ 
                     8.0 &  -3.0 \\
                     10.0 &  3.0 \\
                     13.0 &  4.0 \\
  \end{tabular}
  }
  \end{picture}
}

\put(50,0){
  \begin{picture}(100,250)
    \put(100,220) {The Parametric Curve}

    \put(60,70){\line(2,1){60}}
    \put(120,100){\line(1,2){30}}
    \put(150,160){\line(2,-3){60}}
    \put(210,70){\line(1,3){20}}
    \put(230,130){\line(3,1){30}}

    \put(60,70) {\circle*{2}}
    \put(120,100){\circle*{2}}
    \put(150,160){\circle*{2}}
    \put(210,70){\circle*{2}}
    \put(230,130){\circle*{2}}
    \put(260,140){\circle*{2}}

    \put(20,100){\vector(1,0){250}}
    \put(130,10){\vector(0,1){200}}

    \put(60,60){$(-7.0,-3.0)$}
    \put(120,90){$(-1.0,0.0)$}
    \put(132,165){$(2.0,5.0)$}
    \put(210,60){$(8.0,-3.0)$}
    \put(190,135){$(10.0,3.0)$}
    \put(245,145){$(13.0,4.0)$}
  \end{picture}  
}
\end{picture}
\caption{Defining a Parametric Curve}\label{fig:curve}
\end{center}
\end{figure}

\subsection{The Class Declaration ParametricCurve}
\begin{verbatim}
class ParametricCurve{
public:
  ParametricCurve();
  ParametricCurve(const vector<double>& v);
  ParametricCurve(const CString& file_name);
  bool install(const CString& file_name);
  bool ok();
  float eval(double x);
  CString getFile();
private:
  ParametricCurve& read_xy_file(char *file_name);
  CString file;
  vector<double> v;
  int num_of_elements;
};
\end{verbatim}

\subsection{Public Methods}
\subsubsection{ParametricCurve();}
Default constructor, no parametric curve is defined.
The curve must be defined by method \tt install()\rm.

\begin{description}
    \item [Returns] Nothing.
\end{description} 

\subsubsection{ParametricCurve(const vector<double>\& v);}  
Create a parametric curve defined in vector \tt v\rm.
\begin{description}
    \item [Returns] Nothing.
\end{description} 

\subsubsection{ParametricCurve(const CString\& file\_name);}

Create a parametric curve defined in file \tt file\_name\rm.

\subsubsection{install(const CString\& file\_name);}
Install a parametric cure defined in the file \tt file\_name\rm.
    \begin{description}
       \item [Returns] true if  the parametric curve properly defined,
                       false if not (see method \tt ok())\rm.
    \end{description} 


\begin{description}
    \item [Returns] Nothing.
\end{description} 

\subsubsection{bool ok();} 
Checks if the parametric curve is properly defined. The curve is 
properly defined if each $x$ value has exactly one $y$ value in the file and
the $x$ values are in strictly ascending order.

If you want to define functions that have ``steps'' (i.e, $x$ value
has two $y$ values) you have to define two $x$ values that are 
very close. 
   
    \begin{description}
       \item [Returns] true if  the parametric curve properly defined,
                       false if not. 
    \end{description} 

\subsubsection{float eval(double x);}
Evalute the value of the curve at the point \tt x\rm.
\begin{description}  
    \item [Arguments] for the method.
      \begin{description}
        \item [x] The point where the value of the parametric curve 
                  is evaluated.
       \end{description}
    \item [Returns] The value at the point \tt x\rm.
\end{description} 

\subsubsection{CString getFile()}
\begin{description}
       \item [Returns] The name of the file where the curve is defined
    \end{description}


\subsection{Private Methods}
\subsubsection{ParametricCurve\& read\_xy\_file(char* file\_name);}
Read the file and create the internal representation of the parametric curve.
This method should be reimplemented allowing user to include
comments in the file. (See class \tt Lex\rm).
 \begin{description}
    \item [Arguments] for the method.
      \begin{description}
        \item [file\_name] The file where the parametric curve is defined.
       \end{description}
    \item [Returns] The parametric curve itself.
 \end{description} 

\subsection{Private Data Members}
\begin{description}
 \item [CString file] The name of the file where the parametric curve
                      is defined.
 \item [vector<double> v] Vector containg the internal representation of the
                          parametric curve. 
 \item [int num\_of\_elements]  The length of the vector \tt v\rm.
\end{description}
