\section{Lex}

The class \tt Lex \rm implements a simple tokenizer of a file input 
stream. Each token is reperesented in class \tt Token \rm as a pair
of the type of the token and its string value. In addition
to numbers and identifiers, the following special characters 
are recognised as tokens: \tt :\rm, \tt .\rm, \tt - \rm and \tt /\rm.

Numbers are interpreted either as integers or if a decimal dot
is encounterd as real numbers. The current implementation
cannot recognize scientific exponentials etc., so valid numbers
are, e.g., 12, -12, 12345  and 12.12345.

Identifiers can contain only alphanumeric characters (as returned
by C-library function \it isalnum\rm). Valid identifiers are, e.g.,  
\tt foo\rm, \tt bar \rm and \tt foo1\rm.     

The hash character (\tt \#\rm) is interpreted 
as a beginning of a comment streching to the end of line 
in the file (i.e., all characters from \tt\# \rm to 
the end of line are ignored). This can naturally be discussed
and might change in the future (e.g., parameterize the comment
character(s)). The tokenizer is not a preprocessor for a file,
but this works fine for now; creating relatively simple 
parameter files for the tree model.

\tt Lex \rm will scan an ASCII file, remove all white space and create
a list of \tt Token\rm s from the tokens it recognises.
One speacial token representing the end of file is inserted 
as the last element of the token list.   

\subsection{Using Lex}

In \lignum\ the well defined set of parameters determine
the functioning of the model.  The parameters
can be given in files and the class \tt Lex \rm is used to read 
and tokenize the contents of these files. An  example
of the structure of such file is in Figure \ref{fig:lexfile}.

\begin{figure}[h]
\begin{center}
\begin{picture}(180,30)
%\graphpaper(0,0)(180,30)
\put(10,25){\#A sample file defining 2 parameters}
\put(10,5){
\begin{tabular}{r r l} 

P1 & 1.0 & \#First parameter \\
P2 & 2.0 & \#Second parameter \\

\end{tabular}
}

\end{picture}
\caption{Sample Input File for Lex}\label{fig:lexfile}
\end{center}
\end{figure}

The file in Figure \ref{fig:lexfile} begins with a comment describing
the contents of the file followed by two parameters, their values
and associated comments.
 
Assuming the file name is \tt file.txt \rm the class \tt Lex \rm can 
be used to read and tokenize the file
with the following two operations.

\begin{tabbing}
Tabbing \= Tabbing \= Tabbing \= \kill
\>\>\>\tt Lex lex; \\
\>\>\>\tt lex.scan(``file.txt'');
\end{tabbing}

The first operation declares a lexer. After that the lexer is
told to read the file \tt file.txt.\rm During scanning the file     
and creating the tokens the comments are removed. By denoting a single token 
as a pair $(v,t)$ where $v$ is the string value of token and $t$ the type,
the lexer contains the list of following tokens after processing the file.
\begin{center}
\tt 
[(``P1'',ID),(``1.0'',FLOAT),(``P2'',ID),(``2.0'',FLOAT),(``'',ENDFILE)]
\rm
\end{center}

Now it is relatively easy to parse (interpret) the token list 
with the lexer. For example the following algorithm can be used 
as a framework.

\begin{tabbing}
Tabbing \= Tabbing \= Tabbing \= Tab \= \kill
\>\>\>\tt Token token; \\
\>\>\>\tt token = lex.getToken(); \\  
\>\>\>\tt while (token.getType() != ENDFILE)\{\\
\>\>\>\>\tt //Decide what to do with the token\\
\>\>\>\>\tt lex.getToken();\\
\>\>\>\tt \}
\end{tabbing}

The method \tt getToken() \rm returns one token at a time and
also removes the token from the list. More complex algorithms can
be designed by using methods that only peeks the next token
and puts a token back into the token list.

\subsection{The Class Declaration of Lex}
\begin{verbatim}
class Lex{
public:
  ~Lex();
  Lex& scan(const char *file);
  Token peek();
  Token getToken();
  Lex& putToken(const Token& token);
protected:
  Lex& scanId();
  Lex& scanNum();
  Lex& scanSpecial();
private:
  ifstream ifile;
  DList token_ls;
};
\end{verbatim}

\subsection{Public Methods}

\subsubsection{\~{ }Lex()}

The destructor, delete the list of tokens.
\begin{description}
  \item [Returns]  Nothing.
\end{description}

\subsubsection{Lex\& scan(const char *file);}
Read the file and create the list of
tokens. 
\begin{description}
  \item [Returns] The tokenizer containing now the list of tokens.
\end{description}

\subsubsection{Token peek();}
Return the first token in the token list. The token list is not
changed.
\begin{description}
  \item [Returns] The first token.
\end{description}

\subsubsection{Token getToken();}
Like \tt peek() \rm but the token is removed from
the token list. 
changed.
\begin{description}
  \item [Returns] The first token.
\end{description}

\subsubsection{Lex\& putToken(const Token\& token);}
Insert the token \tt token \rm as the first element in the token list. 
\begin{description}
  \item [Arguments] to the method
    \begin{description}
      \item [token] The token to be inserted.
    \end{description}
   \item [Returns] The tokenizer.
\end{description}

\subsection{Protected Methods}

\subsubsection{Lex\& scanId();}
Generates identifier token using C-library function \tt isalnum\rm. 
However, the first character must be alphabetic. 
\begin{description}
  \item [Returns] The tokenizer.
\end{description}

\subsubsection{Lex\& scanNum();}  
Generates token representing number (integer or real).
\begin{description}
  \item [Returns] The tokenizer.
\end{description}

\subsubsection{Lex\& scanSpecial();}
Generates token for special characters.
\begin{description}
  \item [Returns] The tokenizer.   
\end{description}

\subsection{Private Data Memebers}
\begin{description}
  \item [ifstream ifile] The ASCII file stream.
  \item [DList token\_ls] The list of tokens generated 
   from the file stream \tt ifile\rm. The \tt DList \rm
   is here for historical reasons and might change to the \tt list
  \rm type provided by STL. This change will not affect the visible part
   of the class \tt Lex \rm (i.e., the use of the class). 
\end{description}

\subsection{Token}
A single token in \tt Lex \rm is represented as a pair of its string value and 
token type. The possible token types are enumerated as 
\tt TOKEN\_TYPE\rm.

\subsubsection{The Class Declaration of Token}
\begin{verbatim}
class Token{
public:
  Token():value(""),type(ENDFILE){}
  Token(const CString& token_value, TOKEN_TYPE token_type);
  Token(const Token& token);
  CString getValue()const{return value;}
  TOKEN_TYPE getType()const{return type;}
  Token& operator = (const Token& token);
private:
  CString value;
  TOKEN_TYPE type;
};
\end{verbatim}

\subsubsection{Type Definitions}
The type of the token is represented as enumerated data type 
\tt TOKEN\_TYPE\rm:

\begin{center}
\tt enum TOKEN\_TYPE \{COLON,DOT,FLOAT,ENDFILE,HYPHEN,ID,INT,SLASH\}; 
\rm
\end{center}

The possible values are \tt COLON \rm for ':', \tt DOT \rm for '.', 
\tt FLOAT \rm for real number, \tt ENDFILE \rm for the end of file, 
\tt HYPHEN \rm for '-', \tt ID \rm for identifier, \tt INT \rm for
integer and \tt SLASH \rm for '/'.

\subsubsection{Public Methods}

\subsubsection{Token();}
The default constructor that creates a token representing the
end of file.
\begin{description}
  \item [Returns] Nothing.
\end{description}

\subsubsection{Token(const CString\& value, TOKEN\_TYPE type);}
Constructor that creates token with a known string value and token type.
\begin{description}
  \item [Returns] Nothing.
\end{description}

\subsubsection{Token(const Token\& token);}
Copy constructor.
\begin{description}
  \item [Returns] Nothing.
\end{description}

\subsubsection{CString getValue() const;}
Query the string value of the token.
\begin{description}
  \item [Returns] The string value of the token.
\end{description}

\subsubsection{TOKEN\_TYPE getType() const;}
Query the type of the token.
\begin{description}
  \item [Returns] The type of the token.
\end{description}

\subsubsection{Token\& operator = (const Token\& token);}
The assignment operator.
\begin{description}
  \item [Returns] The token (on the left side of the assignment).
\end{description}

\subsubsection{Private Data Memebers}
\begin{description}
  \item [CString value] The string value of the token.
  \item [TOKEN\_TYPE type]  The type of the token.
\end{description}
